\documentclass{article}
%HEVEA\externalcsstrue
%HEVEA\loadcssfile{../hevea.css}
\usepackage{hevea,ifthen}
\begin{latexonly}
  \makeindex
\end{latexonly}
\newcounter{eks}
\newlength{\ttlength}
\newlength{\eelength}
\newlength{\mmlength}
\begin{document}
%HEVEA\newcommand{\index}[1]{}
\pagenumbering{roman}
\begin{latexonly}
  \tableofcontents
  \parindent 0pt
  \parskip 4mm
\end{latexonly}
\newcommand{\AssignTy}{\verb":"\raisebox{-1pt}{\tt =}$\,$}
\newcommand{\Assign}{\verb":"\raisebox{-1pt}{\tt =}$\,$}
\newcommand{\SwapTy}{\verb":"\raisebox{-1pt}{\tt =}\verb":"}
\newcommand{\ArrowTy}{$\;\rightarrow\;$}
\newcommand{\ExclamTy}{!!}
\newcommand{\PromptTy}{?$\!$?}
\newcommand{\AMP}{\&$\:$}
\newcommand{\UP}[1]{\,\uparrow\!#1}
\newcommand{\REL}[2]{{\tt \begin{tabular}{|l|}\hline \multicolumn{1}{|c|}{#1} \\ \hline #2 \\ \hline \end{tabular}}}
\newcommand{\RER}[2]{{\tt \begin{tabular}{|r|}\hline \multicolumn{1}{|c|}{#1} \\ \hline #2 \\ \hline \end{tabular}}}
\newcommand{\RELR}[3]{{\tt \begin{tabular}{|l|r|}\hline \multicolumn{1}{|c|}{#1} & \multicolumn{1}{c|}{#2} \\ \hline #3 \\ \hline \end{tabular}}}
\newcommand{\RERL}[3]{{\tt \begin{tabular}{|r|l|}\hline \multicolumn{1}{|c|}{#1} & \multicolumn{1}{c|}{#2} \\ \hline #3 \\ \hline \end{tabular}}}
\newcommand{\RERR}[3]{{\tt \begin{tabular}{|r|r|}\hline \multicolumn{1}{|c|}{#1} & \multicolumn{1}{c|}{#2} \\ \hline #3 \\ \hline \end{tabular}}}
\newcommand{\RERLL}[4]{{\tt \begin{tabular}{|r|l|l|}\hline \multicolumn{1}{|c|}{#1} & \multicolumn{1}{c|}{#2} & \multicolumn{1}{c|}{#3} \\ \hline #4 \\ \hline \end{tabular}}}
\newcommand{\RERLR}[4]{{\tt \begin{tabular}{|r|l|r|}\hline \multicolumn{1}{|c|}{#1} & \multicolumn{1}{c|}{#2} & \multicolumn{1}{c|}{#3} \\ \hline #4 \\ \hline \end{tabular}}}
\newcommand{\RELRL}[4]{{\tt \begin{tabular}{|l|r|l|}\hline \multicolumn{1}{|c|}{#1} & \multicolumn{1}{c|}{#2} & \multicolumn{1}{c|}{#3} \\ \hline #4 \\ \hline \end{tabular}}}
\newcommand{\RERRR}[4]{{\tt \begin{tabular}{|r|r|r|}\hline \multicolumn{1}{|c|}{#1} & \multicolumn{1}{c|}{#2} & \multicolumn{1}{c|}{#3} \\ \hline #4 \\ \hline \end{tabular}}}
\newcommand{\RERLLL}[5]{{\tt \begin{tabular}{|r|l|l|l|}\hline \multicolumn{1}{|c|}{#1} & \multicolumn{1}{c|}{#2} & \multicolumn{1}{c|}{#3} & \multicolumn{1}{c|}{#4} \\ \hline #5 \\ \hline \end{tabular}}}
\newcommand{\RERLLRR}[6]{{\tt \begin{tabular}{|r|l|l|r|r|}\hline \multicolumn{1}{|c|}{#1} & \multicolumn{1}{c|}{#2} & \multicolumn{1}{c|}{#3} & \multicolumn{1}{c|}{#4} & \multicolumn{1}{c|}{#5}\\ \hline #6 \\ \hline \end{tabular}}}
\newcommand{\RELLRRR}[6]{{\tt \begin{tabular}{|l|l|r|r|r|}\hline \multicolumn{1}{|c|}{#1} & \multicolumn{1}{c|}{#2} & \multicolumn{1}{c|}{#3} & \multicolumn{1}{c|}{#4} & \multicolumn{1}{c|}{#5}\\ \hline #6 \\ \hline \end{tabular}}}
\newcommand{\RERCRRR}[6]{{\tt \begin{tabular}{|r|c|r|r|r|}\hline \multicolumn{1}{|c|}{#1} & \multicolumn{1}{c|}{#2} & \multicolumn{1}{c|}{#3} & \multicolumn{1}{c|}{#4} & \multicolumn{1}{c|}{#5}\\ \hline #6 \\ \hline \end{tabular}}}
\setcounter{eks}{0}
\settowidth{\ttlength}{\tt x}
\def\eksempel#1{
  \addtocounter{eks}{1}
  \settowidth{\mmlength}{\bf ~Eksempel \arabic{eks}~}
  \setlength{\eelength}{\textwidth}
  \addtolength{\eelength}{-1cm}
  \addtolength{\eelength}{-1pt}
  \addtolength{\eelength}{-\mmlength}
  \ifthenelse{\boolean{hevea}}{
    \@open{div}{class="example"}
    \@open{div}{class="title"}Eksempel \arabic{eks}\@close{div}
  }{
    \rule{1cm}{0.7mm} {\bf ~Eksempel \arabic{eks}~} \rule{\eelength}{0.7mm}
  }
  #1
  \ifthenelse{\boolean{hevea}}{
    \@close{div}
  }{
    \rule{2mm}{3mm}
  }
}
\newcommand{\META}[1]{\mbox{\bf #1}}
\renewcommand{\SS}[1]{\hspace*{#1\ttlength}}
\newcommand{\qind}[1]{#1\index{#1}}
\newpage
\pagenumbering{arabic}

\section{Introduktion}
{\sc Rasmus} er et databasesystem, der er baseret p\aa{} den s\aa{}kaldte
{\em relations\-algebra}. Det underst\o{}tter desuden den {\em funktionelle\/}
programmeringsstil.
        
\subsection{Relationsalgebra}
En {\em relation\/} er en tabel som den f\o{}lgende
\begin{center}
\RERLL{\AA{}rskort{}:Int}{Navn:Text}{Linje:Text}{  
960001 & Trine S\o{}rensen & dat/mat \\
960002 & Kurt Petersen & dat/fys \\
960003 & Knud B\o{}rge Jensen & mat/dat}
\end{center}
Overskriften p\aa{} relationen kaldes for dens {\em \qind{skema}}. 
Det indeholder
et navn for hver s\o{}jle, kaldet en {\em \qind{attribut}}, 
sammen med en
{\em \qind{type}}, der beskriver det lovlige indhold af felterne i s\o{}jlen.
I eksemplet er benyttet typen \verb"Int", der omfatter heltallene, og typen
\verb"Text", der omfatter tegnf\o{}lger.
Hver r\ae{}kke i relationen kaldes et {\em \qind{tupel}}. For hver attribut
rummer det en v\ae{}rdi af den p\aa{}g\ae{}ldende type.

Man f\o{}lger den konvention, at r\ae{}kkef\o{}lgen af s\o{}jlerne
ikke er signifikant. Tilsvarende kan man bytte rundt p\aa{} tuplerne
uden at \ae{}ndre ved relationen. Eksemplet ovenfor kan alts\aa{}
liges\aa{}vel skrives som
\begin{center}
\RELRL{Navn:Text}{\AA{}rskort{}:Int}{Linje:Text}{
Knud B\o{}rge Jensen & 960003& mat/dat\\
Trine S\o{}rensen &960001& dat/mat\\
Kurt Petersen &960002& dat/fys}
\end{center}
Det kr\ae{}ves altid, at der ikke
er to ens r\ae{}kker i en relation; den kan alts\aa{} opfattes som en
{\em m\ae{}ngde\/} af tupler, hvor hvert tupel er en {\em funktion\/}
fra attributnavne til v\ae{}rdier.

Gennem {\sc Rasmus} systemet kan man skabe, modificere og inspicere
relationer. De praktiske aspekter af dette er i detaljer beskrevet 
i en separat brugervejledning.

Det er vigtigt, at man kan {\em kombinere\/} sine oplysninger.
Det sker i {\sc Rasmus} gennem den s\aa{}kaldte
{\em \qind{relationsalgebra}}, hvor man 
anvender {\em operatorer\/} p\aa{} relationer. En operator
tager som operander et antal relationer -- og m\aa{}ske andre v\ae{}rdier --
og opbygger p\aa{} systematisk vis en ny relation.
Der er en traditionel samling af s\aa{}danne operatorer, der vil blive
pr\ae{}senteret i kapitel 3. 

En database skal indholde mere end blot {\em data\/} i form af relationer.
Man skal kunne definere en samling {\em foresp\o{}rgsler},
der tager udgangspunkt i de tilg\ae{}ngelige data. Til dette 
form\aa{}l indeholder {\sc Rasmus} et funktionsbegreb, der
tillader, at man definerer parametriserede relationsudtryk.
Man har faktisk den fulde frihed, der kendes fra funktionelle
programmeringssprog. Dette aspekt pr\ae{}senteres i kapitel 4.

{\sc Rasmus} underst\o{}tter s\aa{}ledes flere slags v\ae{}rdier:
relationer og funktioner, men ogs\aa{} tupler og atomare
v\ae{}rdier er til stede. I kapitel 5 beskrives et antal forskellige
operatorer p\aa{} disse. Deres tilstedev\ae{}relse vil ogs\aa{}
g\o{}re de relationelle operatorer mere slagkraftige.

Gennemgangen af {\sc Rasmus} systemet er hovedsagelig 
drevet af eksempler. I kapitel 6 diskuteres de
generelle begreber, specielt som de relaterer sig til
\o{}vrige programmeringssprog.

\subsection{Fodboldmaterialet}
Som eksempel p\aa{} en database er valgt en del af resultaterne fra den f\o{}rste
s\ae{}son i den danske superliga i fodbold, som blev spillet i for\aa{}ret 1991.
For at begr\ae{}nse st\o{}rrelsen af relationerne i eksemplerne er
turneringen sk\aa{}ret ned til de seks hold, der endte med at v\ae{}re
bedst placerede: Br\o{}ndby, Lyngby, AGF, Frem, OB og AaB. Det er deres
indbyrdes kampe i en dobbeltturnering, der udg\o{}r materialet.

Udover informationen om kampene og deres resultater indeholder data\-bas\-en
ogs\aa{} informationer om en ugentlig ``tipskupon'', som alts\aa{} -- igen
af plads\-hensyn -- kun indeholder tre kampe.

Informationen er opdelt p\aa{} relationer med f\o{}lgende navne og
skemaer:
{\tt
\begin{center}
\begin{tabular}{ll}
\qind{Runde1}: & \RERLL{Id:Int}{HjHold:Text}{UdeHold:Text}{~&~&~} \\ \\
\qind{Runde2}: & \RERLL{Id:Int}{HjHold:Text}{UdeHold:Text}{~&~&~}\\ \\
\qind{Resultater}: & \RERRR{Id:Int}{HjScore:Int}{UdeScore:Int}{~&~&~} \\ \\
\qind{Tips}: & \RERRR{Id:Int}{Uge:Int}{Nr:Int}{~&~&~} \\ \\
\qind{Spilledag}: & \RERL{Kamp:Int}{Dato:Text}{~&~}
\end{tabular}
\end{center}
}
De to f\o{}rste relationer indeholder en angivelse af de spillede kampe;
der er en relation for hver runde i dobbeltturneringen.
Den tredje relation indeholder resultaterne af samtlige kampe; den fjerde angiver
uge for uge udseendet af tipskuponerne; den sidste relation angiver,
p\aa{} hvilke dage de enkelte kampe blev spillet. Indholdet af relationerne er
\newpage
\verb"Runde1:"
\begin{center}
\RERLL{Id:Int}{HjHold:Text}{UdeHold:Text}{
1 &
AGF &
Br\o{}ndby \\
2 &
Frem &
OB \\
3 &
AaB &
Lyngby \\
4 &
OB &
AGF \\
5 &
Br\o{}ndby &
AaB \\
6 &
Lyngby &
Frem \\
7 &
AaB &
AGF \\
8 &
Frem &
Br\o{}ndby \\
9 &
Lyngby &
OB \\
10 &
AGF &
Frem \\
11 &
Br\o{}ndby &
Lyngby \\
12 &
OB &
AaB \\
13 &
Frem &
AaB \\
14 &
Lyngby &
AGF \\
15 &
OB &
Br\o{}ndby}
\end{center}
\verb"Runde2:"
\begin{center}
\RERLL{Id:Int}{HjHold:Text}{UdeHold:Text}{
16 &
AaB &
Frem \\
17 &
AGF &
Lyngby \\
18 &
Br\o{}ndby &
OB \\
19 &
Br\o{}ndby &
AGF \\
20 &
OB &
Frem \\
21 &
AGF &
OB \\
22 &
Lyngby &
AaB \\
23 &
Frem &
AGF \\
24 &
AaB &
Br\o{}ndby \\
25 &
Frem &
Lyngby \\
26 &
AGF &
AaB \\
27 &
Br\o{}ndby &
Frem \\
28 &
OB &
Lyngby \\
29 &
Lyngby &
Br\o{}ndby \\
30 &
AaB &
OB}
\end{center}
\newpage
\verb"Resultater":
\begin{center}
\RERRR{Id:Int}{HjScore:Int}{UdeScore:Int}{
1 &
1 &
2\\
2 &
3 &
0\\
3 &
1 &
2\\
4 &
0 &
0\\
5 &
2 &
2\\
6 &
1 &
1\\
7 &
4 &
1\\
8 &
1 &
1\\
9 &
1 &
0\\
10 &
2 &
1\\
11 &
0 &
3\\
12 &
1 &
1\\
13 &
1 &
2\\
14 &
2 &
1\\
15 &
1 &
1\\
16 &
1 &
2\\
17 &
1 &
1\\
18 &
0 &
0\\
19 &
0 &
0\\
20 &
1 &
1\\
21 &
3 &
1\\
22 &
2 &
0\\
23 &
2 &
2\\
24 &
0 &
2\\
25 &
2 &
1\\
26 &
6 &
1\\
27 &
4 &
1\\
28 &
1 &
1\\
29 &
1 &
1\\
30 &
5 &
1}
\end{center}
\newpage
\verb"Tips:"
\begin{center}
\RERRR{Id:Int}{Uge:Int}{Nr:Int}{
1 &
12 &
2\\
2 &
12 &
3\\
3 &
12 &
1\\
4 &
14 &
3\\
5 &
14 &
1\\
6 &
14 &
2\\
7 &
16 &
1\\
8 &
16 &
2\\
9 &
16 &
3\\
10 &
18 &
1\\
11 &
18 &
2\\
12 &
18 &
3\\
13 &
19 &
1\\
14 &
19 &
2\\
15 &
19 &
3}
\end{center}
\newpage
\verb"Spilledag:"
\begin{center}
\RERL{Kamp:Int}{Dato:Text}{
1 &
316\\
2 &
317\\
3 &
401\\
4 &
324\\
5 &
407\\
6 &
407\\
7 &
414\\
8 &
414\\
9 &
428\\
10 &
421\\
11 &
505\\
12 &
505\\
13 &
512\\
14 &
512\\
15 &
512\\
16 &
516\\
17 &
516\\
18 &
516\\
19 &
520\\
20 &
520\\
21 &
523\\
22 &
526\\
23 &
530\\
24 &
602\\
25 &
602\\
26 &
609\\
27 &
609\\
28 &
619\\
29 &
623\\
30 &
623}
\end{center}

%% \newpage
%% \section{Rasmus systemet}
%% {\sc Rasmus} systemet, der benyttes gennem {\sc Emacs},
%% er beskrevet i en speciel brugervejledning.
%% N\aa{}r det startes, s\aa{} ser man to buffere.
%% I den ene skrives {\em udtryk}, hvis resultater vises
%% i den anden. Man kan altid navngive
%% og gemme det sidste resultat. I denne note vil vi med 
%% notationen \fbox{\fbox{\tt Navn}} angive, at det sidste resultat er
%% gemt under navnet {\tt Navn}. 

\newpage
\section{Relationsudtryk}
Udover at skabe og udskrive relationer, kan man naturligvis ogs\aa{}
kombinere relationer med hinanden. Dette foreg\aa{}r efter
{\em relationsalgebraens\/}\index{relationsalgebra}
principper, og den letteste m\aa{}de at forklare
dette p\aa{} er via en analogi med de s\ae{}dvanlige regningsarter
p\aa{} heltal. Et aritmetisk heltalsudtryk som
$$7+5$$
betyder, at operatoren $+$ skal anvendes p\aa{} operanderne 7 og 5.
Resultatet heraf (udtrykkets v\ae{}rdi) er som bekendt 12, og en s\aa{}dan
v\ae{}rdi kan man s\aa{} g\o{}re et eller andet videre ved, fx kan man
gemme den.

I {\sc Rasmus} kan relationer optr\ae{}de i relationsudtryk p\aa{} samme
m\aa{}de, som tal kan optr\ae{}de i aritmetiske udtryk. Operatorerne er
nogle andre, men betydningen er i princippet den samme: resultatet af 
at anvende en
relationsoperator er en relation -- pr\ae{}cis ligesom resultatet af
at anvende en heltalsoperator er et heltal. 

Syntaksen for relationsoperatorer i {\sc Rasmus} er nogenlunde den samme
som for de s\ae{}dvanlige regningsarter, idet vi ogs\aa{} benytter
fx $+$ og $*$, men nu med en anden betydning.

\subsection{Union}
Man kan danne en relation, som er en foreningsm\ae{}ngde af to andre
relationer. Dette sker ved hj\ae{}lp af den s\aa{}kaldte
{\em union}-operator, som vi udtrykker ved hj\ae{}lp af \verb"+" p\aa{}
denne m\aa{}de
$$ \META{relation}_{\mbox{\bf 1}}\mbox{\tt ~+~}\META{relation}_{\mbox{\bf 2}}$$
Det kr\ae{}ves, at de to operander har samme skema.
Udtrykkets resultat er en ny relation, som man kan bruge overalt, hvor
en relation ellers er tilladt, fx som operand til en ny relationsoperator.
\newpage
\eksempel{
Skab en relation best\aa{}ende af samtlige tupler fra de to
turneringsrunder
$$ \mbox{\tt Kampe := Runde1 + Runde2} $$
Hvis vi gemmer resultatet af dette udtryk under navnet \fbox{\fbox{\tt Kampe}} (som vist i koden ovenfor)
vil relationen {\tt \qind{Kampe}} have f\o{}lgende indhold
\begin{small}
\begin{center}
\RERLL{Id:Int}{HjHold:Text}{UdeHold:Text}{
 1 & AGF      & Br\o{}ndby \\
 2 & Frem     & OB       \\
 3 & AaB      & Lyngby   \\
 4 & OB       & AGF      \\
 5 & Br\o{}ndby & AaB      \\
 6 & Lyngby   & Frem     \\
 7 & AaB      & AGF      \\
 8 & Frem     & Br\o{}ndby \\
 9 & Lyngby   & OB       \\
10 & AGF      & Frem     \\
11 & Br\o{}ndby & Lyngby   \\
12 & OB       & AaB      \\
13 & Frem     & AaB      \\
14 & Lyngby   & AGF      \\
15 & OB       & Br\o{}ndby \\
16 & AaB      & Frem     \\
17 & AGF      & Lyngby   \\
18 & Br\o{}ndby & OB       \\
19 & Br\o{}ndby & AGF      \\
20 & OB       & Frem     \\
21 & AGF      & OB       \\
22 & Lyngby   & AaB      \\
23 & Frem     & AGF      \\
24 & AaB      & Br\o{}ndby \\
25 & Frem     & Lyngby   \\
26 & AGF      & AaB      \\
27 & Br\o{}ndby & Frem     \\
28 & OB       & Lyngby   \\
29 & Lyngby   & Br\o{}ndby \\
30 & AaB      & OB       
}
\end{center}
\end{small}
}
\newpage
Union-operatorens m\aa{}de at kombinere relationer
p\aa{} kan siges at v\ae{}re ``lodret'', idet resultatrelationen bliver
l\ae{}ngere (i alt fald ikke kortere) end den l\ae{}ngste af operanderne.

\subsection{Join}
Man kan ogs\aa{} kombinere relationer ``vandret'', det vil sige, s\aa{}
resultatet bliver bredere (i alt fald ikke smallere) end den
bredeste af operanderne. Operatoren hertil hedder {\em \qind{join}\/} og
optr\ae{}der i udtryk i form af \verb"*" p\aa{} f\o{}lgende m\aa{}de
$$ \META{relation}_{\mbox{\bf 1}}\mbox{\tt ~*~}\META{relation}_{\mbox{\bf 2}}$$
Resultatet af dette udtryk er en ny relation, hvis attributnavne udg\o{}res
af alle de attributnavne, der findes i de to operander tilsammen.
Eventuelle attributnavne, der optr\ae{}der begge steder, kommer dog kun med
en enkelt gang; det kr\ae{}ves, at de to operander er enige om
typerne for s\aa{}danne f\ae{}lles attributnavne.

Tuplerne i den nye relation dannes ved at kombinere alle par af tupler fra
de to relationer, som har samme v\ae{}rdier p\aa{} attributter med samme navn.

Join-operatoren kan give en tom relation som resultat, hvis
der ikke er to tupler med samme v\ae{}rdi p\aa{} de f\ae{}lles attributter.
Hvis de to operander ikke har nogen f\ae{}lles
attributter, vil resultatet indeholde alle mulige kombinationer
af deres tupler.

\eksempel{
Skab en relation, der indeholder samtlige kampe og deres resultater
$$ \mbox{\tt Turnering := Kampe * Resultater} $$
Hvis resultatet af udtrykket gemmes under navnet \fbox{\fbox{\tt Turnering}} 
er v\ae{}rdien af {\tt \qind{Turnering}} som f\o{}lger
\begin{center}
\begin{small}
\RERLLRR{Id:Int}{HjHold:Text}{UdeHold:Text}{HjScore:Int}{UdeScore:Int}{
 1 & AGF      & Br\o{}ndby &       1 &        2 \\ 
 2 & Frem     & OB       &       3 &        0 \\ 
 3 & AaB      & Lyngby   &       1 &        2 \\ 
 4 & OB       & AGF      &       0 &        0 \\ 
 5 & Br\o{}ndby & AaB      &       2 &        2 \\ 
 6 & Lyngby   & Frem     &       1 &        1 \\ 
 7 & AaB      & AGF      &       4 &        1 \\ 
 8 & Frem     & Br\o{}ndby &       1 &        1 \\ 
 9 & Lyngby   & OB       &       1 &        0 \\ 
10 & AGF      & Frem     &       2 &        1 \\ 
11 & Br\o{}ndby & Lyngby   &       0 &        3 \\ 
12 & OB       & AaB      &       1 &        1 \\ 
13 & Frem     & AaB      &       1 &        2 \\ 
14 & Lyngby   & AGF      &       2 &        1 \\ 
15 & OB       & Br\o{}ndby &       1 &        1 \\ 
16 & AaB      & Frem     &       1 &        2 \\ 
17 & AGF      & Lyngby   &       1 &        1 \\ 
18 & Br\o{}ndby & OB       &       0 &        0 \\ 
19 & Br\o{}ndby & AGF      &       0 &        0 \\ 
20 & OB       & Frem     &       1 &        1 \\ 
21 & AGF      & OB       &       3 &        1 \\ 
22 & Lyngby   & AaB      &       2 &        0 \\ 
23 & Frem     & AGF      &       2 &        2 \\ 
24 & AaB      & Br\o{}ndby &       0 &        2 \\ 
25 & Frem     & Lyngby   &       2 &        1 \\ 
26 & AGF      & AaB      &       6 &        1 \\ 
27 & Br\o{}ndby & Frem     &       4 &        1 \\ 
28 & OB       & Lyngby   &       1 &        1 \\ 
29 & Lyngby   & Br\o{}ndby &       1 &        1 \\ 
30 & AaB      & OB       &       5 &        1}
\end{small}
\end{center}
}

\subsection{Select}
Det er ovenfor vist, at relationer kan g\o{}res l\ae{}ngere og bredere,
s\aa{} det er ikke s\ae{}rligt overraskende, at de ogs\aa{} kan g\o{}res
kortere og smallere.

En m\aa{}de at g\o{}re dem kortere p\aa{} er at udv\ae{}lge de tupler,
der opfylder en given betingelse. Hertil anvendes den s\aa{}kaldte
{\em \qind{select}}-operator, som vi skriver p\aa{} f\o{}lgende m\aa{}de
$$ \META{relation} \mbox{\tt ~?~(} \META{betingelse} ) $$
Resultatet er en relation, som indeholder netop de tupler, der opfylder
betingelsen.

\eksempel{
Skab en relation best\aa{}ende af de kampe i turneringen, hvor
hjemmeholdet vandt
$$ \mbox{\verb"Turnering ? (\#.HjScore > \#.UdeScore)"} $$
V\ae{}rdien af dette udtryk er
\begin{center}
\begin{small}
\RERLLRR{Id:Int}{HjHold:Text}{UdeHold:Text}{HjScore:Int}{UdeScore:Int}{
 2 & Frem     & OB      &       3 &        0 \\ 
 7 & AaB      & AGF     &       4 &        1 \\ 
 9 & Lyngby   & OB      &       1 &        0 \\ 
10 & AGF      & Frem    &       2 &        1 \\ 
14 & Lyngby   & AGF     &       2 &        1 \\ 
21 & AGF      & OB      &       3 &        1 \\ 
22 & Lyngby   & AaB     &       2 &        0 \\ 
25 & Frem     & Lyngby  &       2 &        1 \\ 
26 & AGF      & AaB     &       6 &        1 \\ 
27 & Br\o{}ndby & Frem    &       4 &        1 \\ 
30 & AaB      & OB      &       5 &        1}
\end{small}
\end{center}
}

Betingelsen \verb"#.HjScore > #.UdeScore" er et eksempel p\aa{} en
{\em simpel\/} \qind{be\-ting\-else}. Vi skal senere se, at man kan skrive langt
mere avancerede be\-ting\-elser i {\sc Rasmus}, og vi vil ogs\aa{} forklare
hvad \verb"#" betyder. For n\ae{}r\-v\ae{}r\-en\-de er det tilstr\ae{}kkeligt
at forst\aa{}, at ovenn\ae{}vnte betingelse udv\ae{}lger de tup\-ler,
hvor v\ae{}rdien af attributten \verb"HjScore" er st\o{}rre end
v\ae{}rdien af attributten \verb"UdeScore".

\subsection{Project}
Hvis relationer skal g\o{}res smallere, anvendes den
s\aa{}kaldte {\em \qind{project}}-ope\-ra\-tor, som vi skriver som
$$ \META{relation} \mbox{\verb" |+ "} \META{attributliste} $$
Resultatet er den del af operandrelationen, der udg\o{}res af attributterne
i den angivne \qind{attributliste}, medens de \o{}vrige attributter
``smides v\ae{}k''. Bem\ae{}rk, at project-operatoren samtidigt kan
g\o{}re relationen kortere, da to tupler fra operanden kan blive ens,
n\aa{}r nogle af attributterne fjernes.

\eksempel{
Skab en relation, der for hver kamp indeholder det antal m\aa{}l, der
er scoret af udeholdet
$$ \mbox{\verb"Resultater |+ Id,UdeScore"} $$
V\ae{}rdien af udtrykket er
\begin{center}
\begin{small}
\RERR{Id:Int}{UdeScore:Int}{
 1 &        2\\
 2 &        0\\
 3 &        2\\
 4 &        0\\
 5 &        2\\
 6 &        1\\
 7 &        1\\
 8 &        1\\
 9 &        0\\
10 &        1\\
11 &        3\\
12 &        1\\
13 &        2\\
14 &        1\\
15 &        1\\
16 &        2\\
17 &        1\\
18 &        0\\
19 &        0\\
20 &        1\\
21 &        1\\
22 &        0\\
23 &        2\\
24 &        2\\
25 &        1\\
26 &        1\\
27 &        1\\
28 &        1\\
29 &        1\\
30 &        1}
\end{small}
\end{center}
}

Vi kunne have opn\aa{}et samme resultat ved at bruge det alternative
udtryk
$$ \mbox{\verb"Resultater |- HjScore"} $$
idet man ved varianten \verb"|-" angiver de attributter, der skal fjernes.

\subsection{Sammensatte udtryk}
Alle relationsudtryk i de foreg\aa{}ende eksempler har v\ae{}ret
``flade'', i den forstand, at operanderne til de forskellige
relationsoperatorer har v\ae{}ret simple relationer. Operander kan imidlertid
selv v\ae{}re udtryk, hvis op\-er\-an\-der igen kan v\ae{}re udtryk, og s\aa{}
videre.

P\aa{} samme m\aa{}de som man i et aritmetisk udtryk kan skrive fx
$$ (7+5)*3 $$
kan man i relationsalgebraen skrive fx
$$ \mbox{\verb"(Runde1 + Runde2) |+ Id"} $$
Dette sammensatte udtryk beregner 
identiteterne p\aa{} samtlige kampe i dobbeltturneringen.

\subsection{Rename}
Foruden join og project er der endnu en operator, som kan siges at virke
``vandret''. Det er den s\aa{}kaldte {\em \qind{rename}}-operator, der er
n\o{}dvendig, hvis en relations attributnavne ikke er
hensigtsm\ae{}ssige for en efterf\o{}lgende brug. Antag fx, at
der \o{}nskes foretaget en kombination af relationerne \verb"Runde1" og
\verb"Spilledag", s\aa{}ledes at kampe, der optr\ae{}der i begge
relationer, bliver repr\ae{}senteret ved de samlede oplysninger om dem.
Man kan ikke umiddelbart benytte join, da identitetsattributterne har
forskellige navne; i \verb"Runde1" hedder den \verb"Id", og i
\verb"Spilledag" hedder den \verb"Kamp". 
Hvis vi join'ede de to relationer ville resultatet blive en relation
med 450 tupler, idet alle tupler i \verb"Runde1" ville blive
parret med alle tupler i \verb"Spilledag". Dette er ikke hvad
vi \o{}nsker, og det er derfor n\o{}dvendigt at kunne omd\o{}be
attributnavne. Det g\o{}res ved hj\ae{}lp af {\em rename}-operatoren,
som anvendes p\aa{} f\o{}lgende m\aa{}de
$$ \META{relation} \mbox{\tt ~[~} \META{\qind{navneparliste}} \mbox{\tt ~]~} $$
I \META{navneparliste} st\aa{}r de udskiftninger af navne, som vi
\o{}nsker at foretage, p\aa{} formen
\META{gammelt navn} \verb"<-" \META{nyt navn}, adskilte af kommaer.

\eksempel{
V\ae{}rdien af udtrykket
\mbox{\verb"Spilledag [ Kamp <- Id ]"} er som f\o{}lger
\begin{center}
\begin{small}
\RERL{Id:Int}{Dato:Text}{
 1 & 316 \\
 2 & 317 \\
 3 & 401 \\
 4 & 324 \\
 5 & 407 \\
 6 & 407 \\
 7 & 414 \\
 8 & 414 \\
 9 & 428 \\
10 & 421 \\
11 & 505 \\
12 & 505 \\
13 & 512 \\
14 & 512 \\
15 & 512 \\
16 & 516 \\
17 & 516 \\
18 & 516 \\
19 & 520 \\
20 & 520 \\
21 & 523 \\
22 & 526 \\
23 & 530 \\
24 & 602 \\
25 & 602 \\
26 & 609 \\
27 & 609 \\
28 & 619 \\
29 & 623 \\
30 & 623 
}
\end{small}
\end{center}
}

Nu kan vi udtrykke nyttige kombinationer af \verb"Runde1" og
\verb"Spilledag".

\eksempel{
Skab en relation best\aa{}ende af datoerne for
AGF's hjem\-me\-kam\-pe i f\o{}rste runde af turneringen
$$\mbox{\verb:((Runde1 ? (\#.HjHold="AGF"))*(Spilledag[Kamp<-Id])) |+ Dato:} $$
V\ae{}rdien er
\begin{center}
\REL{Dato:Text}{
316\\
421}
\end{center}
}

\subsection{Difference}
Vi har ogs\aa{} en s\aa{}kaldt {\em \qind{difference}}-operator, som bruges til
at finde de tupler, der optr\ae{}der i een relation, men ikke i en anden.
Udseendet er
$$ \META{relation}_{\mbox{\bf 1}} \mbox{\verb" - "} \META{relation}_{\mbox{\bf 2}} $$
Det kr\ae{}ves som ved union, at de to operander har samme skema.

\eksempel{
Find ud af, hvilke hold der ikke har vundet p\aa{} udebane
\begin{center}
\begin{tabular}{l}
\verb"(Turnering |+ UdeHold) - "\\
\verb"((Turnering ? (\#.HjScore<\#.UdeScore)) |+ UdeHold)"
\end{tabular}
\end{center}
V\ae{}rdien af udtrykket er
\begin{center}
\REL{UdeHold:Text}{
AGF \\
OB
}
\end{center}
}

Bem\ae{}rk, at det her er vigtigt, at relationer ikke indeholder dubletter.
V\ae{}rdien af det f\o{}rste deludtryk ovenfor,
\verb"Turnering |+ UdeHold", er
\begin{center}
\REL{UdeHold:Text}{
AGF \\
Frem\\
AaB \\
OB\\
Br\o{}ndby \\
Lyngby
}
\end{center}
Vi ser, at hvert hold kun optr\ae{}der en enkelt gang, uanset hvor mange
kampe, det har spillet. Dette skyldes som tidligere n\ae{}vnt, at
relationer er {\em m\ae{}ngder}, og dette er en vigtig foruds\ae{}tning
for, at difference-operatoren giver mening.

N\aa{}r {\sc Rasmus} s\aa{}ledes er udstyret med de to m\ae{}ngde-operatorer
union og difference for foreningsm\ae{}ngde og m\ae{}ngdedifferens, s\aa{}
ville det v\ae{}re at forvente, at der ogs\aa{} var en operator for
\qind{f\ae{}llesm\ae{}ngde}. Det er der ogs\aa{} -- men kun indirekte -- fordi
det er nemt at se, at hvis to relationer har samme skema, s\aa{} vil
join beregne deres f\ae{}llesm\ae{}ngde. Resultatet vil indeholde
de tupler, der har samme v\ae{}rdi p\aa{} alle attributter, og det er jo
netop de f\ae{}lles tupler.

\subsection{Indskudte udtryk}
Vi slutter dette kapitel med et lidt mere kompliceret eksempel, hvor
vi bruger \verb"Tips"-relationen for f\o{}rste gang.

\eksempel{
Find nummeret p\aa{} den kamp p\aa{} tipskuponen for uge 18, hvor
AGF var med
\begin{center}
\begin{tabular}{l}
\verb:(((Kampe*(Tips ? (\#.Uge=18))) ? (\#.UdeHold="AGF")) +: \\
\verb: ((Kampe*(Tips ? (\#.Uge=18))) ? (\#.HjHold="AGF"))) |+ Nr:
\end{tabular}
\end{center}
Resultatet bliver
\begin{center}
\RER{Nr:Int}{
1}
\end{center}
}

Som vi skal se i et senere kapitel, kan dette udtryk simplificeres en smule,
men alligevel kan vi allerede nu se, at n\aa{}r relationsudtryk bliver
store, s\aa{} bliver der behov for at kunne opskrive dem p\aa{} en mere
hensigtsm\ae{}ssig m\aa{}de, end vi har set hidtil. Til den ende
har {\sc Rasmus} f\o{}lgende midlertidige navngivningsmekanisme,
et s\aa{}kaldt {\em \qind{indskudt udtryk}}
{\tt
\begin{center}
\begin{tabular}{ll}
(+ & val $\META{id}_{\mbox{\bf 1}}$ = $\META{relation}_{\mbox{\bf 1}}$ \\
   & val $\META{id}_{\mbox{\bf 2}}$ = $\META{relation}_{\mbox{\bf 2}}$ \\
   & $\vdots$ \\
   & val $\META{id}_{\mbox{\bf n}}$ = $\META{relation}_{\mbox{\bf n}}$ \\
   & in \META{relation}\\
+) &
\end{tabular}
\end{center}
}
Dette udtryk bevirker, at man under udregningen af det sidste udtryk
\META{relation} kan referere til v\ae{}rdierne af
$\META{relation}_{\mbox{\bf 1}}$ til $\META{relation}_{\mbox{\bf n}}$ 
ved hj\ae{}lp
af navnene $\META{id}_{\mbox{\bf 1}}$, \ldots, $\META{id}_{\mbox{\bf n}}$. Resultatet af det samlede
\verb"(+ ... +)"- udtryk er den s\aa{}ledes udregnede 
v\ae{}rdi af \META{relation}.
Navnene $\META{id}_{\mbox{\bf 1}}$, \ldots, $\META{id}_{\mbox{\bf n}}$ 
er {\em lokale\/} og kan ikke
refereres uden for \verb"(+ ... +)"- udtrykket. I
udtrykket $\META{relation}_{\mbox{\bf i}}$ kan man benytte navnene
$\META{id}_{\mbox{\bf 1}}$, \ldots, $\META{id}_{\mbox{\bf i$-$1}}$.

\eksempel{
Udtrykket i eksempel 8 kan ogs\aa{} skrives p\aa{} f\o{}lgende facon
\begin{center}
\begin{tabular}{ll}
\verb"(+" & \verb:val X = Kampe*(Tips ? (\#.Uge=18)): \\
   & \verb:in ((X ? (\#.HjHold="AGF"))+(X ? (\#.UdeHold="AGF"))) |+ Nr:\\
\verb"+)" &
\end{tabular}
\end{center}
}

\newpage
\section{Funktioner}
{\sc Rasmus} er ogs\aa{} udstyret med en {\em funktionsmekanisme},
der dels kan bruges til at {\em parametrisere\/} relationsudtryk
og dels til at opbygge {\em biblioteker\/} af nyttige udtryk.

Betragt igen eksempel 8. Det er sandsynligt, at hvis man er interesseret
i nummeret p\aa{} AGF's tipskamp i uge 18, s\aa{} er man formentlig
ogs\aa{} interesseret i numrene p\aa{} andre holds tipskampe i andre
uger.

\eksempel{Skriv en funktion, der, givet et ugenummer og navnet p\aa{} et
hold, returnerer en relation, der indeholder holdets kampnummer p\aa{}
den p\aa{}\-g\ae{}ld\-en\-de tipskupon
\begin{center}
\begin{tabular}{l}
\verb"FindNr :=" \\
\SS{2}\verb"func (uge:Int, hold:Text) -> (Rel)"\\
\SS{4}\verb:(+ val X = Kampe*(Tips ? (\#.Uge=uge)): \\
\SS{7}\verb:in ((X ? (\#.HjHold=hold))+(X ? (\#.UdeHold=hold))) |+ Nr:\\
\SS{4}\verb"+)"\\
\SS{2}\verb"end"
\end{tabular}
\end{center}
}

Hvis v\ae{}rdien af dette udtryk gemmes under navnet \fbox{\fbox{\verb"FindNr"}}\index{FindNr}
vil f\o{}lgende anvendelse af funktionen
$$ \mbox{\verb:FindNr(18,"AGF"):} $$
give samme resultat som udtrykkene i eksempel 8 og 9.
Udtrykket
\begin{center}
\begin{tabular}{l}
\verb"func (uge:Int, hold:Text) -> (Rel)"\\
\verb"  ..."\\
\verb"end"
\end{tabular}
\end{center}
er en s\aa{}kaldt {\em \qind{funktionsdefinition}}, medens udtrykket
$$ \mbox{\verb:FindNr(18,"AGF"):} $$
er en {\em \qind{funktionsanvendelse}}. Funktionens {\em hoved}
$$ \mbox{\verb$func (uge:Int, hold:Text) -> (Rel)$} $$
angiver at der er tale om en funktion med to {\em parametre}, henholdsvis
\verb"uge" og \verb"hold", hvis {\em typer\/} er henholdsvis
\verb"Int" og \verb"Text"; man kan desuden se, at funktionens resultat er
af type \verb"Rel".
Navnene \verb"uge" og \verb"hold" er {\em formelle\/} \qind{parametre}. 

Funktionen er beregnet til at blive anvendt med forskellige
{\em aktuelle\/} para\-metre; i eksemplet ovenfor er \verb"18" og
\verb:"AGF": s\aa{}ledes aktuelle parametre. Man kan naturligvis
ogs\aa{} bruge andre aktuelle parametre, som i anvendelsen
$$ \mbox{\verb:FindNr(19,"Lyngby"):} $$
Betydningen af en funktionsanvendelse er, at det udtryk, der
udg\o{}r funktionens {\em \qind{krop}\/} -- her udtrykket
\verb"(+ ... +)" -- beregnes med de formelle parametre erstattet med
v\ae{}rdierne af de aktuelle parametre.

{\em Typenavnene\/} \verb"Int" og \verb"Text" angiver, at den f\o{}rste
aktuelle parameter skal v\ae{}re af typen \verb"Int", det vil sige et heltal,
medens den anden skal v\ae{}re af typen \verb"Text", som betegner
m\ae{}ngden af tegnf\o{}lger. Typeangivelsen \verb"(Rel)" p\aa{}
h\o{}jresiden af \verb"->" fort\ae{}ller, at resultatet er af type
\verb"Rel", som betegner m\ae{}ngden af relationer.

I de n\ae{}ste to eksempler definerer vi to funktioner, der er nyttige, hvis
vi skal konstruere en tipskupon for en given uge. Vi antager, at vi f\o{}rst
har anvendt {\sc Rasmus} systemet til at konstruere f\o{}lgende
tre relationer:
{\tt
\begin{center}
Et:~\begin{tabular}{|c|} \hline Tegn \\\hline 1 \\\hline\end{tabular}~~~~
Kryds:~\begin{tabular}{|c|} \hline Tegn \\\hline X \\\hline\end{tabular}~~~~
To:~\begin{tabular}{|c|} \hline Tegn \\\hline 2 \\\hline\end{tabular}
\end{center}
}

\eksempel{
Skriv en funktion, der for en given uge konstruerer relationen
\begin{center}
\begin{small}
\RELLRRR{{\tt HjHold:Text}}{{\tt UdeHold:Text}}{{\tt HjScore:Int}}{{\tt UdeScore:Int}}{{\tt Nr:Int}}{~&~&~&~&~}
\end{small}
\end{center}
best\aa{}ende af kampene p\aa{} den p\aa{}g\ae{}ldende uges tipskupon
\begin{center}
\begin{tabular}{l}
\verb"Strip :="\\
\SS{2}\verb"func (uge:Int) -> (Rel)"\\
\SS{4}\verb"((Turnering*Tips) ? (\#.Uge=uge)) |- Id,Uge"\\
\SS{2}\verb"end"
\end{tabular}
\end{center}
}

Hvis denne funktion gemmes under navnet \fbox{\fbox{\verb"Strip"}}, s\aa{} kan vi
benytte den til at l\o{}se problemet.\index{Strip}

\eksempel{
Skriv en funktion, der for en given uge konstruerer tipskuponen
\begin{center}
\begin{tabular}{l}
\verb"Kupon :="\\
\SS{2}\verb"func (ugenr:Int) -> (Rel)"\\
\SS{4}\verb"(((Strip(ugenr) ? (\#.HjScore < \#.UdeScore))*To) +"\\
\SS{5}\verb"((Strip(ugenr) ? (\#.HjScore = \#.UdeScore))*Kryds) +"\\
\SS{5}\verb"((Strip(ugenr) ? (\#.HjScore > \#.UdeScore))*Et)"\\
\SS{4}\verb") |- HjScore,UdeScore"\\
\SS{2}\verb"end"
\end{tabular}
\end{center}
}

Hvis denne nu gemmes under navnet \fbox{\fbox{\verb"Kupon"}} har vi det
\o{}nskede resultat. Fx vil anvendelsen \verb"Kupon(18)" have\index{Kupon}
v\ae{}rdien
\begin{center}
\RERLLL{Nr:Int}{HjHold:Text}{UdeHold:Text}{Tegn:Text}{
2 & Br\o{}ndby & Lyngby  & 2 \\ 
3 & OB       & AaB     & X  \\
1 & AGF      & Frem    & 1}
\end{center}

\newpage
\section{Generelle udtryk}
De udtryk, vi hidtil har besk\ae{}ftiget os med, har n\ae{}sten alle
v\ae{}ret {\em relationsudtryk}, det vil sige, udtryk hvis resultater er
relationer, og hvor de indg\aa{}ende operatorer er relationsoperatorerne
union, join og s\aa{} videre. {\sc Rasmus} tillader imidlertid udtryk af
flere forskellige typer, herunder {\em boolske\/} udtryk og
{\em aritmetiske\/} udtryk, hvis resultater er henholdvis
sandhedsv\ae{}rdier og heltal. En betingelse i en select-operator er
s\aa{}ledes i det generelle tilf\ae{}lde et boolsk udtryk. 

\subsection{Typer og konstanter} 
Indtil nu har vi stiftet bekendskab med f\o{}lgende typer\index{type}
\begin{center}
\begin{tabular}{rl}
\verb"Int": & heltal \\
\verb"Float": & komma tal \\
\verb"Text": & tegnf\o{}lger \\
\verb"Rel": & relationer \\
\verb"Func": & funktioner
\end{tabular}
\end{center}
Der findes endnu to typer i {\sc Rasmus}
\begin{center}
\begin{tabular}{rl}
\verb"Bool": & sandhedsv\ae{}rdier \\
\verb"Tup": & tupler
\end{tabular}
\end{center}\index{Int}\index{Float}\index{Text}\index{Bool}\index{Rel}\index{Tup}\index{Func}
Typerne \verb"Int", \verb"Float", \verb"Text" og \verb"Bool", 
hvis v\ae{}rdier er dem der kan st\aa{} i de
enkelte felter i relationer, kaldes {\em atomare\/} og kan under et\index{atomar}
angives med f\ae{}llesbetegnelsen \verb"Atom".  Den totale samling af de seks 
typer kan angives med f\ae{}llesbetegnelsen \verb"Any".

De atomare typer har de s\ae{}dvanlige {\em konstanter}, det vil sige\index{konstant}
\begin{center}
\begin{tabular}{rl}
\verb"Int": & \ldots, \verb"-1", \verb"0", \verb"1", \verb"2", \ldots\\
\verb"Float": & \ldots, \verb"-1.2", \verb"0.0", \verb"1e-7", \verb"2e3", \ldots\\
\verb"Bool": & \verb"true", \verb"false"\\
\verb"Text": & \verb:"dette er en konstant af type Text":
\end{tabular}
\end{center}
At konstanter af type \verb"Text" omgives med \verb:": og \verb:": skyldes,
at det er n\o{}dvendigt at kunne skelne {\em navnet\/} \verb"AGF" fra
tegnf\o{}lgen \verb:"AGF":, som er tegnet \verb"A" efterfulgt af tegnet
\verb"G" efterfulgt af tegnet \verb"F".

Typen \verb"Tup" har ogs\aa{} en slags konstanter, idet et udtryk af formen
$$ \mbox{\verb"tup("$\META{attribut}_{\mbox{\bf 1}}$\verb":"$\META{v\ae{}rdi}_{\mbox{\bf 1}}$\verb","
\ldots\verb","$\META{attribut}_{\mbox{\bf n}}$\verb":"$\META{v\ae{}rdi}_{\mbox{\bf n}}$\verb")"} $$
angiver et tupel med n felter, hvis attributnavne er
$\META{attribut}_{\mbox{\bf 1}}$, 
\ldots, $\META{attribut}_{\mbox{\bf n}}$, og hvis tilh\o{}rende
v\ae{}rdier er $\META{v\ae{}rdi}_{\mbox{\bf 1}}$, \dots, $\META{v\ae{}rdi}_{\mbox{\bf n}}$.
Man kan ogs\aa{} konstruere en slags konstante relationer; hvis \verb"t"
nemlig er et tupel, s\aa{} er
$$ \mbox{\verb"rel(t)"} $$
en relation best\aa{}ende af det enkelte tupel \verb"t".

\eksempel{
Skab en relation best\aa{}ende af de tre tipstegn
$$\mbox{\verb$rel(tup(Tegn:"1"))+rel(tup(Tegn:"X"))+rel(tup(Tegn:"2"))$ }$$
V\ae{}rdien af dette udtryk er
\begin{center}
\REL{Tegn:Text}{
1 \\
X \\
2}
\end{center}
}

Vi kan nu skrive funktionen \verb"Kupon" i eksempel 12 p\aa{} f\o{}lgende 
alternative m\aa{}de. Bem\ae{}rk, at vi kan bruge tidligere
\verb"Val"-definitioner i senere \verb"val"-definitioner.
\newpage
\eksempel{
\begin{center}
\begin{tabular}{l}
\verb$func (ugenr:Int) -> (Rel)$\\
\SS{2}\verb$(+ val Strip = $\\
\SS{9}\verb$func (uge:Int) -> (Rel)$\\
\SS{11}\verb$((Turnering*Tips) ? (\#.Uge=uge)) |- Id,Uge$\\
\SS{9}\verb"end"\\
\SS{5}\verb$val X = Strip(ugenr)$\\
\SS{5}\verb$val Et = $\\
\SS{9}\verb$(X ? (\#.HjScore>\#.UdeScore))*rel(tup(Tegn:"1"))$\\
\SS{5}\verb$Val Kryds = $\\
\SS{9}\verb$(X ? (\#.HjScore=\#.UdeScore))*rel(tup(Tegn:"X"))$\\
\SS{5}\verb$val To = $\\
\SS{9}\verb$(X ? (\#.HjScore<\#.UdeScore))*rel(tup(Tegn:"2"))$\\
\SS{5}\verb$in (Et+Kryds+To) |- HjScore,UdeScore$\\
\SS{2}\verb$+)$\\
\verb"end"
\end{tabular}
\end{center}
}

\subsection{Sammensatte betingelser}
Vi illustrerer brugen af boolske udtryk i de f\o{}lgende tre
eksempler. I det f\o{}rste eksempel skriver vi en alternativ udgave af
eksempel 8.

\eksempel{
Find nummeret p\aa{} den kamp p\aa{} tipskuponen for uge 18, hvor
AGF var med
\begin{center}
\begin{tabular}{l}
\verb$((Kampe*(Tips ? (\#.Uge=18))) ?$\\
\SS{4}\verb$((\#.Udehold="AGF") or (\#.HjHold="AGF))) |+ Nr$
\end{tabular}
\end{center}
}

Vi kan f\aa{} brug for mere komplicerede boolske udtryk.
\newpage
\eksempel{
Skab en relation best\aa{}ende af de kampe, der fik en vinder, og hvor
taberen ikke scorede
\begin{center}
\begin{tabular}{l}
\verb$Turnering ? (((\#.UdeScore > 0) and (\#.HjScore = 0)) or$\\
\SS{13}\verb$((\#.UdeScore = 0) and (\#.HjScore > 0)))$
\end{tabular}
\end{center}
V\ae{}rdien af udtrykket er
\begin{center}
\begin{small}
\RERLLRR{Id:Int}{HjHold:Text}{UdeHold:Text}{HjScore:Int}{UdeScore:Int}{
 2 & Frem     & OB       &       3 &        0\\
 9 & Lyngby   & OB       &       1 &        0\\
11 & Br\o{}ndby & Lyngby   &       0 &        3\\
22 & Lyngby   & AaB      &       2 &        0\\
24 & AaB      & Br\o{}ndby &       0 &        2}
\end{small}
\end{center}
}

Man kan ogs\aa{} f\aa{} brug for betingelser, hvor der indg\aa{}r 
beregninger p\aa{} fx tuplernes attributv\ae{}rdier.

\eksempel{
Find de kampe, hvor hjemmeholdet vandt med mere end to m\aa{}l
\begin{center}
\begin{tabular}{l}
\verb$Turnering ? (\#.HjScore - \#.UdeScore > 2)$
\end{tabular}
\end{center}
}

\subsection{Forall og tupeludtryk}
I det f\o{}lgende skal vi beskrive mere pr\ae{}cist, hvordan et generelt
boolsk udtryk ser ud, men f\o{}rst skal vi introducere den
s\aa{}kaldte {\em \qind{forall}}-operator, som er en relationsoperator, der
giver mulighed for at konstruere relationer ved at ``regne p\aa{}''
{\em enkelttupler\/} fra andre relationer. Operatoren har f\o{}lgende
udseende
\begin{center}
{\tt !(}\META{relation}{\tt ):} \META{udtryk}
\end{center}
hvor \META{udtryk} skal v\ae{}re af type \verb"Rel". Betydningen af
et forall-udtryk er, at tuplerne i \META{relation} et for et
``uds\ae{}ttes'' for den mekanisme, der angives af \META{udtryk},
hvorefter resultaterne heraf forenes til en relation, som s\aa{} er
resultatet af hele forall-udtrykket.

Betragt f\o{}lgende eksempel, hvor vi \o{}nsker at udregne,
hvad vi kan kalde underholdningsv\ae{}rdien af de enkelte fodboldkampe,
defineret som antal m\aa{}l scoret af hjemmeholdet plus to gange
antal m\aa{}l scoret af udeholdet.

\eksempel{
Udvid relationen {\tt Turnering} med en attribut, som for hver kamp
indeholder dennes underholdningsv\ae{}rdi
\begin{center}
\begin{tabular}{l}
\verb"!(Turnering): rel(\# << tup(UV: \#.HjScore+2*\#.UdeScore))"
\end{tabular}
\end{center}
V\ae{}rdien af udtrykket er
\begin{center}
\begin{small}
\RERCRRR{Id:Int}{...}{HjScore:Int}{UdeScore:Int}{UV: Int}{
 6 &&       1 &        1 &  3 \\ 
25 &&       2 &        1 &  4 \\ 
10 &&       2 &        1 &  4 \\ 
23 &&       2 &        2 &  6 \\ 
12 &&       1 &        1 &  3 \\ 
30 &&       5 &        1 &  7 \\ 
11 &&       0 &        3 &  6 \\ 
29 &&       1 &        1 &  3 \\ 
 1 &&       1 &        2 &  5 \\ 
19 &&       0 &        0 &  0 \\ 
 2 &&       3 &        0 &  3 \\ 
20 &&       1 &        1 &  3 \\ 
13 &&       1 &        2 &  5 \\ 
16 &&       1 &        2 &  5 \\ 
15 &&       1 &        1 &  3 \\ 
18 &&       0 &        0 &  0 \\ 
 5 &&       2 &        2 &  6 \\ 
24 &&       0 &        2 &  4 \\ 
 8 &&       1 &        1 &  3 \\ 
27 &&       4 &        1 &  6 \\ 
14 &&       2 &        1 &  4 \\ 
17 &&       1 &        1 &  3 \\ 
 9 &&       1 &        0 &  1 \\ 
28 &&       1 &        1 &  3 \\ 
 4 &&       0 &        0 &  0 \\ 
21 &&       3 &        1 &  5 \\ 
 3 &&       1 &        2 &  5 \\ 
22 &&       2 &        0 &  2 \\ 
 7 &&       4 &        1 &  6 \\ 
26 &&       6 &        1 &  8}
\end{small}
\end{center}
}

Dette eksempel illustrerer flere ting. For det f\o{}rste kan vi se, at
udtrykket p\aa{} h\o{}jresiden af \verb":" er af formen
\begin{center}
\verb"rel(# << t)"
\end{center}
hvor \verb"t" er et tupel med en enkelt attribut, \verb"UV".
V\ae{}rdien af denne attribut er
\begin{center}
\verb"#.HjScore+2*#.UdeScore"
\end{center}
der udregnes p\aa{} f\o{}lgende m\aa{}de.
Symbolet \verb"#" er {\em standardnavnet\/} p\aa{} hvad vi kan kalde
det {\em l\o{}bende\/} tupel i forall-udtrykket, og
\verb"#.HjScore" angiver derfor v\ae{}rdien af attributten
\verb"HjScore" i dette tupel. Givet \verb"#" er v\ae{}rdien af
\verb"#.HjScore+2*#.UdeScore" s\aa{}ledes lig med underholdnings\-v\ae{}rdien
af \verb"#". I udtrykket \verb"rel(# << t)" angiver \verb"t" derfor
et tupel med eneste attribut \verb"UV", hvis v\ae{}rdi er
underholdningsv\ae{}rdien af \verb"#". Udtrykket \verb"# << t"
angiver nu resultatet af at anvende {\em tupeloperatoren\/}\index{tupeloperator}
\verb"<<" p\aa{} de to tupler \verb"#" og \verb"t". Operatoren
\verb"<<" er en {\em udvidelses- og overskrivningsoperator}, der virker
som f\o{}lger. Resultatet af\index{udvidelse}
\begin{center}
\verb"t1 << t2"
\end{center}
er det tupel, der f\aa{}s ved f\o{}rst at kopiere \verb"t1",
dern\ae{}st, hvis der er f\ae{}lles attributter i \verb"t1" og \verb"t2",
at erstatte v\ae{}rdierne i \verb"t1" med de tilsvarende fra \verb"t2",
og endeligt at udvide med de attributter fra \verb"t2", der ikke findes
i \verb"t1". F\o{}lgende eksempel illustrerer dette.

\eksempel{
Resultatet af tupeludtrykket
\begin{center}
\verb"tup(Dig:1, Mig:2) << tup(Mig:3, ViTo:4)"
\end{center}
er lig med \verb"tup(Dig:1, Mig:3, ViTo:4)".
}

Vi kan nu se, at udtrykket \verb"# << t" ovenfor netop angiver et tupel
fra \verb"Turnering" udvidet med den p\aa{}g\ae{}ldende kamps
underholdningsv\ae{}rdi. Derfor angiver hele udtrykket
\verb"rel(# << t)" en {\em relation\/} indeholdende netop dette tupel.
Resultatet af hele forall-udtrykket i eksempel 18 best\aa{}r af
foreningsm\ae{}ngden af alle disse relationer, det vil sige,
af samtlige kampe i turneringen udvidet med en
underholdningsv\ae{}rdi-attribut.

Der findes endnu en tupeloperator, nemlig {\em eliminationsoperatoreren\/}
$\backslash$, der bruges som f\o{}lger\index{elimination}
\begin{center}
\verb"t"$\backslash$\verb"A"
\end{center}
hvor \verb"t" er et tupel og \verb"A" er en attribut. Resultatet er
en kopi af tuplet \verb"t", med en eventuel \verb"A"-attribut fjernet.
Vi kan illustrere denne operator ved hj\ae{}lp af f\o{}lgende alternative
udgave af eksempel 4.

\eksempel{
Skab en relation, der for hver kamp indeholder det antal m\aa{}l, der
er scoret af udeholdet
\begin{center}
\verb"!(Resultater): rel(\# "$\backslash$\verb" HjScore)"
\end{center}
}

Med introduktionen af forall-operatoren og den dertil h\o{}rende
forklaring p\aa{} betydningen af \verb"#" skulle det nu v\ae{}re
klart, at \verb"#" spiller pr\ae{}cist samme rolle i select-udtryk,
som den g\o{}r i forall-udtryk. I select-operatoren ``genneml\o{}bes''
relationens tupler p\aa{} samme vis, og de tupler, der opfylder
betingelsen ``smutter forbi'', medens de \o{}vrige ``filtreres fra''.

\subsection{Betingede udtryk}\index{betinget udtryk}
{\sc Rasmus} er ogs\aa{} udstyret med {\em betingede udtryk}, der ser
ud som f\o{}lger
\begin{center}
\begin{tabular}{l}
\verb"if "$\META{betingelse}_{\mbox{\bf 1}}$\verb" -> "$\META{udtryk}_{\mbox{\bf 1}}$\\
\verb" & "$\META{betingelse}_{\mbox{\bf 2}}$\verb" -> "$\META{udtryk}_{\mbox{\bf 2}}$\\
\verb" "$\vdots$\\
\verb" & "$\META{betingelse}_{\mbox{\bf n}}$\verb" -> "$\META{udtryk}_{\mbox{\bf n}}$\\
\verb"fi"
\end{tabular}
\end{center}
Udtrykket kan bruges til at v\ae{}lge mellem alternativer, som
beskrevet af f\o{}lgende eksempel.

\eksempel{
Skriv en funktion, der givet ugenummer og navnet p\aa{} et hold
beregner en relation, der fort\ae{}ller, om det p\aa{}g\ae{}ldende
hold er p\aa{} ude- eller hjemmebane i den p\aa{}g\ae{}ldende uge
\begin{center}
\begin{tabular}{l}
\verb"func (uge:Int, hold:Text) -> (Rel)"\\
\SS{2}\verb"(+ val X = Kampe*(Tips ? (\#.Uge=uge))"\\
\SS{5}\verb"in"\\
\SS{5}\verb@if |X ? (\#.HjHold=hold)| > 0 -> rel(tup(Bane:"hjemme"))@\\
\SS{6}{\tt \&~}\verb@true -> rel(tup(Bane:"ude"))@\\
\SS{5}\verb"fi"\\
\SS{2}\verb"+)"\\
\verb"end"
\end{tabular}
\end{center}
}

Vi har i ovenst\aa{}ende eksempel benyttet os af operatoren \verb"|R|",
som angiver st\o{}rrelsen af relationen \verb"R", det vil sige,
antallet af tupler i \verb"R".

Med introduktionen af tupeloperatorer og forall-operatoren kan vi nu
skrive f\o{}lgende alternative udgave af eksempel 14. Bem\ae{}rk, at
vi her bruger \verb"Val"-mekanismen i indskudte udtryk til at
introducere {\em korte\/} navne for attributudtrykkene
\verb"#.HjScore" og \verb"#.UdeScore".
\eksempel{
\begin{center}
\begin{tabular}{l}
\verb$func (ugenr:Int) -> (Rel)$\\
\SS{2}\verb$(+ val Strip = $\\
\SS{9}\verb$func (uge:Int) -> (Rel)$\\
\SS{11}\verb$((Turnering*Tips) ? (\#.Uge=uge)) |- Id,Uge$\\
\SS{9}\verb"end"\\
\SS{5}\verb$val X = Strip(ugenr)$\\
\SS{5}\verb$in !(X): (+ val H = \#.HjScore$\\
\SS{17}\verb$val U = \#.UdeScore$\\
\SS{17}\verb$in rel((\#$$\backslash$\verb$HjScore)$$\backslash$\verb$UdeScore <<$\\
\SS{27}\verb$tup(Tegn: if U<H -> "1"$\\
\SS{38}{\tt \&}\verb$ U=H -> "X"$\\
\SS{38}{\tt \&}\verb$ U>H -> "2"$\\
\SS{37}\verb$fi$\\
\SS{30}\verb")"\\
\SS{23}\verb")"\\
\SS{14}\verb"+)"\\
\SS{2}\verb$+)$\\
\verb"end"
\end{tabular}
\end{center}
}

Som det fremg\aa{}r af det foreg\aa{}ende, kan man b\aa{}de navngive
og gemme relationer og funktioner, og man kan ogs\aa{} give
dem lokale navne ved hj\ae{}lp af \verb"Val"-mekanismen i indskudte
udtryk, eller ved hj\ae{}lp af formelle parametre i funktioner.
Der er imidlertid ikke kun relationer og funktioner, der nyder denne
bev\aa{}genhed -- alle {\sc Rasmus} typer kan benyttes p\aa{}
tilsvarende vis. Dette betyder, at udtryk af de forskellige typer
kan konstrueres og komponeres p\aa{} helt ensartet vis, hvilket ogs\aa{}
kan udtrykkes som, at ``alle typer er f\o{}rsteklasses borgere''.

Som n\ae{}vnt har {\sc Rasmus} typerne
\verb"Bool", \verb"Int", \verb"Float", \verb"Text", \verb"Tup", \verb"Rel" og \verb"Func".
Ethvert udtryk har en type, nemlig typen af den v\ae{}rdi, der
bliver resultatet. 

\subsection{Tekstudtryk}
Vi har tidligere set eksempler p\aa{} udtryk af samtlige typer
-- dog har vi for \verb"Text"-udtrykkene kun set konstanter som \verb$"AGF"$.
Der findes imid\-lertid ogs\aa{} \verb"Text"-operatorer, som illustreret ved
f\o{}lgende eksempel p\aa{} en funktion, der s\ae{}tter \aa{}rstal
p\aa{} de datoangivelser, der findes i \verb"Spilledag" relationen.

\eksempel{
\begin{center}
\begin{tabular}{l}
\verb$func (${\tt aar}\verb$:Text, dato:Text) -> (Text)$\\
\SS{2}\verb$if |dato|=4 -> ${\tt aar}\verb$++dato$\\
\SS{3}{\tt \&}\verb$ |dato|=3 -> ${\tt aar}\verb$++"0"++dato$\\
\SS{2}\verb$fi$\\
\verb"end"
\end{tabular}
\end{center}
}

Hvis vi kalder denne funktion for \fbox{\fbox{\tt LangDato}} er
v\ae{}rdien af anvendelsen
\begin{center}
\verb$Langdato("96","325")$
\end{center}
lig med \verb"960325". Operatoren \verb"++" sammens\ae{}tter 
{\em konkatenerer\/} alts\aa{}\index{konkatenation}
to tegnf\o{}lger, og \verb"|t|" returnerer l\ae{}ngden af en tegnf\o{}lge
\verb"t". Man kan ogs\aa{} udtage delf\o{}lger af tegnf\o{}lger. V\ae{}rdien
af udtrykket\index{delf\o{}lge} 
\begin{center}
\verb$"325"(0..2)$
\end{center}
er s\aa{}ledes \verb$"32"$, idet udtrykket \verb"t(i..j)" i almindelighed
betegner den delf\o{}lge af \verb"t", der starter i det \verb"i"'te tegn
(nummereret fra 0) og ender i det \verb"(j-1)"'te tegn. F\o{}lgende
eksempel illustrerer nu brugen af hj\ae{}lpefunktionen \verb"LangDato"
i et forall-udtryk.

\eksempel{
Skab en relation magen til \verb"Spilledag", men hvor 
datoangivelserne nu er komplette
\begin{center}
\verb$!(Spilledag): rel(\# << tup(Dato: LangDato("91",\#.Dato)))$
\end{center}
V\ae{}rdien af udtrykket er
\begin{center}
\begin{small}
{\tt
\begin{tabular}{|r|l|}
\hline
Kamp:Int & Dato:Text \\\hline
   3 & 910401 \\
  19 & 910520 \\ 
  20 & 910520 \\ 
  10 & 910421 \\ 
  23 & 910530 \\ 
  24 & 910602 \\ 
  25 & 910602 \\ 
  13 & 910512 \\ 
  14 & 910512 \\ 
  15 & 910512 \\ 
   7 & 910414 \\ 
   8 & 910414 \\ 
   4 & 910324 \\ 
  21 & 910523 
\end{tabular}
~~~~~~~
\begin{tabular}{|r|l|}
  ~~~~~~11 & 910505~~~ \\ 
  12 & 910505 \\ 
   1 & 910316 \\ 
  29 & 910623 \\ 
  30 & 910623 \\ 
   5 & 910407 \\ 
   6 & 910407 \\ 
   2 & 910317 \\ 
  16 & 910516 \\ 
  17 & 910516 \\ 
  18 & 910516 \\ 
  22 & 910526 \\ 
   9 & 910428 \\ 
  26 & 910609 \\ 
  27 & 910609 \\ 
  28 & 910619 \\\hline
\end{tabular} 
}
\end{small}
\end{center}
}

\subsection{Skema- og typecheck}
N\aa{}r man skriver funktioner, der overf\o{}rer relationer som parametre,
s\aa{} kender man selvsagt ikke skemaerne for de relationer, der optr\ae{}der
som {\em aktuelle\/} parametre, n\aa{}r funktionen anvendes. {\sc Rasmus}
indeholder derfor en facilitet, der tillader at sp\o{}rge om en relation har
et givet attributnavn. F\o{}lgende eksempel illustrerer dette.

\eksempel{
Skriv en funktion, der for en given {\tt Kamp}-relation 
og et givet hold returnerer de kampe, hvor det p\aa{}g\ae{}ldende hold
er p\aa{} hjemmebane.
\begin{center}
\begin{tabular}{l}
\verb"func (K:Rel, hold:Text) -> (Rel)"\\
\SS{2}\verb"if has(K,HjHold) -> K ? (\#.HjHold=hold)"\\
\SS{3}{\tt \&}\verb" true -> zero"\\
\SS{2}\verb"fi"\\
\verb"end"
\end{tabular}
\end{center}
}

Udtrykket \verb"has(R,A)" giver \verb"true", hvis relationen
\verb"R" har attributten \verb"A" og ellers \verb"false".
Som eksemplet viser, kan man bruge dette til at
undg\aa{} at l\o{}be ind i fejl under eventuel anvendelse
af en funktion p\aa{} et argument, der ikke har det rigtige skema.
I eksemplet returneres den tomme relation, angivet af konstanten
\verb"zero", hvis argumentet ikke ``passer'', men man kunne 
naturligvis ogs\aa{} returnerer en relation, der indeholder en
passende advarsel.

Man kan ogs\aa{} unders\o{}ge, om en relation har den rigtige type.
Hvis denne facilitet anvendes i eksempel 25, f\aa{}r funktionen f\o{}lgende
udseende.
\begin{center}
\begin{tabular}{l}
\verb"func (K:Rel, hold:Text) -> (Rel)"\\
\SS{2}\verb"if has(K,HjHold) and is-Text(K,HjHold) -> K ? (#.HjHold=hold)"\\
\SS{3}{\tt \&}\verb" true -> zero"\\
\SS{2}\verb"fi"\\
\verb"end"
\end{tabular}
\end{center}
Udtrykket \verb"is-Text(R,A)" unders\o{}ger om attributten \verb"A" i
relationen \verb"R" har type Text. Der findes tilsvarende funktioner
\verb"is-Int" og \verb"is-Bool". 

\subsection{Aggregeringsfunktioner}
Som det sidste eksempel i denne ``fodbold-Odyss\'{e}'' viser vi nu,
hvordan man beregner en relation med skema
\begin{center}
\RELR{Hold:Text}{Points:Int}{~&~}
\end{center}
der for hver af de seks deltagende klubber angiver, hvor mange point
klubben fik i turneringen. Til brug herfor skal vi introducere
{\em aggregeringsfunktioner}, som tillader at foretage diverse
beregninger p\aa{} samtlige v\ae{}r\-di\-er af en given attribut i en relation.
F\o{}lgende simple eksempel illustrerer dette.\index{aggregering}

\eksempel{
Angiv, hvor mange m\aa{}l der blev scoret p\aa{} udebane i turneringen
\begin{center}
\verb"add(Resultater,UdeScore)"
\end{center}
V\ae{}rdien af udtrykket er \verb"33".
}

Aggregeringsfunktionen \verb"add(R,A)" adderer v\ae{}rdierne af
attributten \verb"A" i relationen \verb"R". De \o{}vrige
aggregeringsfunktioner er
\begin{center}
\verb"count(R,A), mult(R,A), max(R,A), min(R,A)"
\end{center}
og de henholdsvis t\ae{}ller, multiplicerer, maksimerer og minimerer
\verb"A"-v\ae{}rdi\-er\-ne i \verb"R".\index{count}\index{mult}\index{max}\index{min}\index{add}
\newpage
\eksempel{
Skab en relation, som for hver kamp angiver hvor mange point de to hold
fik i den p\aa{}g\ae{}ldende kamp
\begin{center}
\begin{tabular}{l}
\verb"EnkeltPoints :="\\
\SS{2}\verb"!(Turnering):"\\
\SS{6}\verb"(+ val H = \#.HjScore"\\
\SS{9}\verb"val U = \#.UdeScore"\\
\SS{9}\verb"in if H<U -> rel(tup(Id:\#.Id, Hold:\#.UdeHold, Points:2))"\\
\SS{13}{\tt \&}\verb" H=U -> rel(tup(Id:\#.Id, Hold:\#.UdeHold, Points:1)) +"\\
\SS{22}\verb"rel(tup(Id:\#.Id, Hold:\#.HjHold, Points:1))"\\
\SS{13}{\tt \&}\verb" H>U -> rel(tup(Id:\#.Id, Hold:\#.HjHold, Points:2))"\\
\SS{12}\verb"fi"\\
\SS{6}\verb"+)"
\end{tabular}
\end{center}
V\ae{}rdien er
\begin{center}
\begin{small}
{\tt
\begin{tabular}{|r|l|r|}
\hline
Id:Int & Hold:Text & Points:Int \\ \hline
 6 & Frem     &      1 \\ 
 6 & Lyngby   &      1 \\ 
25 & Frem     &      2 \\ 
10 & AGF      &      2 \\ 
23 & AGF      &      1 \\ 
23 & Frem     &      1 \\ 
12 & AaB      &      1 \\ 
12 & OB       &      1 \\ 
30 & AaB      &      2 \\ 
11 & Lyngby   &      2 \\ 
29 & Br\o{}ndby &      1 \\ 
29 & Lyngby   &      1 \\ 
 1 & Br\o{}ndby &      2 \\ 
19 & AGF      &      1 \\ 
19 & Br\o{}ndby &      1 \\ 
 2 & Frem     &      2 \\ 
20 & Frem     &      1 \\ 
20 & OB       &      1 \\ 
13 & AaB      &      2 \\ 
16 & Frem     &      2 \\ 
15 & Br\o{}ndby &      1
\end{tabular}
}
~~~~~~~
{\tt
\begin{tabular}{|r|l|r|}
~~~~15 & OB~~~~~~~       &      ~~~~~~~~~1 \\ 
18 & OB       &      1 \\ 
18 & Br\o{}ndby &      1 \\ 
 5 & AaB      &      1 \\ 
 5 & Br\o{}ndby &      1 \\ 
24 & Br\o{}ndby &      2 \\ 
 8 & Br\o{}ndby &      1 \\ 
 8 & Frem     &      1 \\ 
27 & Br\o{}ndby &      2 \\ 
14 & Lyngby   &      2 \\ 
17 & Lyngby   &      1 \\ 
17 & AGF      &      1 \\ 
 9 & Lyngby   &      2 \\ 
28 & Lyngby   &      1 \\ 
28 & OB       &      1 \\ 
 4 & AGF      &      1 \\ 
 4 & OB       &      1 \\ 
21 & AGF      &      2 \\ 
 3 & Lyngby   &      2 \\ 
22 & Lyngby   &      2 \\ 
 7 & AaB      &      2 \\ 
26 & AGF      &      2 \\\hline
\end{tabular}
}
\end{small}
\end{center}
}

Hvis vi gemmer denne relation under navnet \fbox{\fbox{\tt EnkeltPoints}}\index{EnkeltPoints}
f\aa{}r vi det \o{}n\-ske\-de resultat p\aa{} f\o{}lgende m\aa{}de.
\eksempel{
\begin{center}
\begin{tabular}{l}
\verb$(+ val AGF = EnkeltPoints ? (\#.Hold="AGF")$\\
\SS{3}\verb$val OB = EnkeltPoints ? (\#.Hold="OB")$\\
\SS{3}\verb$val AaB = EnkeltPoints ? (\#.Hold="AaB")$\\
\SS{3}\verb$val Lyngby = EnkeltPoints ? (\#.Hold="Lyngby")$\\
\SS{3}\verb$val Br${\tt \o{}}\verb$ndby = EnkeltPoints ? (\#.Hold="Br${\tt \o{}}\verb$ndby")$\\
\SS{3}\verb$val Frem = EnkeltPoints ? (\#.Hold="Frem")$\\
\SS{3}\verb$in rel(tup(Hold:"AGF", Points:add(AGF,Points)) +$\\ 
\SS{6}\verb$rel(tup(Hold:"OB", Points:add(OB,Points)) +$\\ 
\SS{6}\verb$rel(tup(Hold:"AaB", Points:add(AaB,Points)) +$\\ 
\SS{6}\verb$rel(tup(Hold:"Lyngby", Points:add(Lyngby,Points)) +$\\ 
\SS{6}\verb$rel(tup(Hold:"Br${\tt \o{}}\verb$ndby", Points:add(Br${\tt \o{}}\verb$ndby,Points)) +$\\ 
\SS{6}\verb$rel(tup(Hold:"Frem", Points:add(Frem,Points))$\\ 
\verb$+)$
\end{tabular}
\end{center}
V\ae{}rdien er
\begin{center}
\RELR{Hold:Text}{Points:Int}{
Lyngby   &    14\\
AGF      &    10\\
OB       &     6\\
AaB      &     8\\
Frem     &    10\\
Br\o{}ndby &    12}
\end{center}
}

\newpage
\subsection{Factor}
Denne m\aa{}de at udregne turneringsresultatet er ikke netop elegant, og hvis
\verb"Hold"-attributten indeholdt flere end seks forskellige v\ae{}rdier,
eller -- endnu v\ae{}rre -- hvis vi ikke kendte v\ae{}rdierne p\aa{}
forh\aa{}nd, s\aa{} er fremgangsm\aa{}den slet ikke brugbar. Derfor
indeholder {\sc Rasmus} en mere generel facilitet, den s\aa{}kaldte
{\em factor}-operator, som tillader at udf\o{}re denne type\index{factor}
beregninger p\aa{} mere hensigtsm\ae{}ssig vis. I dette tilf\ae{}lde
ville beregningen se ud som f\o{}lger.

\eksempel{
\begin{center}
\begin{tabular}{l}
\verb"!(EnkeltPoints)|Hold: rel(\# << Tup(Points:add(@(1),Points)))"
\end{tabular}
\end{center}
}

Lad os se, hvordan ovenst\aa{}ende fungerer. Argumentet til factor-be\-reg\-ning\-en
er relationen \verb"EnkeltPoints". Desuden angiver man et attributnavn, i dette
tilf\ae{}lde \verb"Hold". P\aa{} baggrund heraf beregnes en r\ae{}kke {\em faktorpar} af
formen $(t_1,r_1),(t_2,r_2),\ldots,(t_k,r_k)$. Tuplerne 
$t_1,t_2,\ldots,t_k$ er netop de, der findes i relationen
\verb"EnkeltPoints|+Hold". Relationen $r_i$ er netop
\verb"(EnkeltPoints*rel("$t_i$\verb"))|-Hold". Det vil sige, der g\ae{}lder
f\o{}lgende ligning:
$$\mbox{\tt Enkeltpoints} = \sum_{i=1}^k \mbox{\tt rel(}t_i\mbox{\tt )*}r_i$$ 
For hvert af disse faktorpar beregnes h\o{}jresiden af factor-udtrykket med
\verb"#" erstattet med $t_i$ og \verb"@(1)" erstattet med $r_i$. Det
endelige resultat er foreningen af disse delresultater. For ovenst\aa{}ende
eksempel har vi f\o{}lgende seks faktorpar.

\begin{center}
\begin{tabular}{r@{~~~~~}l}
\fbox{\tt Lyngby} & 
\RERR{Id:Int}{Points:Int}{
6 & 1\\
11 & 2\\
29 & 1\\
14 & 2\\
17 & 1\\
9 & 2\\
28 & 1\\
3 & 2\\
22 & 2}\\
~ & ~\\
\fbox{\tt AGF} & 
\RERR{Id:Int}{Points:Int}{
10 & 2\\
23 & 1\\
19 & 1\\
17 & 1\\
4 & 1\\
21 & 2\\
26 & 2}\\
~ & ~\\
\fbox{\tt OB} & 
\RERR{Id:Int}{Points:Int}{
12 & 1\\
20 & 1\\
15 & 1\\
18 & 1\\
28 & 1\\
4 & 1}\\
~ & ~\\
\fbox{\tt AaB} & 
\RERR{Id:Int}{Points:Int}{
12 & 1\\
30 & 2\\
13 & 2\\
5 & 1\\
7 & 2}
\end{tabular}
\end{center}
\begin{center}
\begin{tabular}{r@{~~~~~}l}
\fbox{\tt Frem} & 
\RERR{Id:Int}{Points:Int}{
6 & 1\\
25 & 2 \\
23 & 1\\
2 & 2 \\
20 & 1\\
16 & 2\\
8 & 1}\\
~ & ~\\
\fbox{\tt Br\o{}ndby} & 
\RERR{Id:Int}{Points:Int}{
29 & 1\\
1 & 2\\
19 & 1\\
15 & 1\\
18 & 1\\
5 & 1\\
24 & 2\\
8 & 1\\
27 & 2}
\end{tabular}
\end{center}
Det skulle nu v\ae{}re klart, hvorledes vi beregner det \o{}nskede resultat.

Et factor-udtryk kan involvere flere attributnavne. I s\aa{} fald indeholder
faktorparrets tupel alle disse attributter og relationerne er tilsvarende
smallere. 

\eksempel{
Skab en relation, der for hver mulig score fort\ae{}ller, hvor mange kampe
der fik denne score.
\begin{center}
\verb"!(Turnering)|HjScore,UdeScore: rel(\# << tup(Antal: |@(1)|))"
\end{center}
V\ae{}rdien af udtrykket er
\begin{center}
\RERRR{HjScore:Int}{UdeScore:Int}{Antal:Int}{
0 & 0 & 3\\
1 &0 & 1\\
1 & 1 & 8\\
2 & 0 & 1\\
0 & 2 & 1\\
1 & 2 & 4\\
3 & 0 & 1\\
2 & 1 & 3\\
0 & 3 & 1\\
2 & 2 & 2\\
3 & 1 & 1\\
4 & 1 & 2\\
5 & 1 & 1\\
6 & 1 & 1}
\end{center}
}

Det skulle nu let kunne ses, at forall-operatoren blot er et specialtilf\ae{}lde
af factor, hvor man angiver {\em samtlige\/} attributnavne. I s\aa{} tilf\ae{}lde
best\aa{}r faktorparrene blot af alle relationens tupler parret med
relationen \verb"one", der indeholder et enkelt tupel med tomt skema
(som ikke bidrager med nogen interessant information).

Factor-operationen er dog endnu mere generel, idet man ogs\aa{} kan angive
flere relationer. I s\aa{} fald skal de angivne attributnavne tilh\o{}re
skemaerne for samtlige de indg\aa{}ende relationer. Nu f\aa{}r man ikke blot
nogle faktor{\em par}, men sekvenser best\aa{}ende af et tupel og en delrelation for
hver indg\aa{}ende relation. Delrelationen for den $i$'te relation ben\ae{}vnes
\verb"@(i)". Hvis vi betragter et generelt factor-udtryk:
\begin{center}
{\tt !(R$_1$,\ldots,R$_m$)|n$_1$,\ldots,n$_l$: \ldots}
\end{center}
hvor den $i$'te af $k$ faktorsekvenser er $(t_i,r_{i1},\ldots,r_{im})$, s\aa{}
g\ae{}lder f\o{}lgende matrixligning:
$$ (\mbox{\tt R}_1,\ldots,\mbox{\tt R}_m) = 
   (\mbox{\tt rel(}t_1\mbox{\tt )},\ldots,\mbox{\tt rel(}t_k\mbox{\tt )})
   \left(\begin{array}{llll}
   r_{11} & r_{12} & \cdots & r_{1m} \\
   r_{21} & r_{22} & \cdots & r_{2m} \\
   \vdots & \vdots & & \vdots \\
   r_{k1} & r_{k2} & \cdots & r_{km} 
   \end{array}\right)
$$
Med denne udvidelse er factor nu s\aa{} generel, at man kan n\o{}jes med
den som eneste operation (sammen med krydsprodukt af relationer med
disjunkte skemaer).

\subsection{Oversigt}
Som tidligere n\ae{}vnt er alle typer i {\sc Rasmus}
``f\o{}rste klasses borgere''. Det betyder, at v\ae{}rdier af alle typer
kan b\aa{}de navngives og gemmes, og at udtryk af alle typer
kan v\ae{}re parametre til og resultater af funktioner. I det f\o{}lgende
giver vi en mere systematisk fremstilling af udseendet af de forskellige
udtrykstyper. Vi betegner udtryk af type 
\verb"Int", \verb"Float", \verb"Bool", \verb"Text", \verb"Tup", \verb"Rel"
og \verb"Func" med henholdsvis $\cal I$, $\cal D$, $\cal B$, $\cal X$, $\cal T$, 
$\cal R$ og $\cal F$. Et udtryk af en atomar type betegnes $\cal A$.
Et udtryk af en {\em ordnet\/} type betegnes $\cal O$; alle typer\index{ordnet type}
p\aa{}n\ae{}r \verb"Func" er ordnede og kan dermed sammenlignes med
\verb"=", \verb"<>", \verb"<", \verb">", \verb"<=" og \verb">=".
Et udtryk af en vilk\aa{}rlig type angives med $\cal U$.
Et \verb"val"-, parameter- eller
attributnavn betegnes med \verb"n".
\newpage
Udtryk af type \verb"Bool" $(\cal B)$:\index{Bool}
\begin{center}
\begin{tabular}{|l|l|}
\hline
Betegnelse & Syntaks \\\hline
\qind{konstant} & \verb"true", \verb"false" \\
\qind{standardv\ae{}rdi} & \verb"?-Bool" \\
\qind{navn} & \verb"n" \\
\qind{attribut} & $\cal T$\verb".n" \\
\qind{sammenligning} & ${\cal O}_1$\verb"="${\cal O}_2$, ${\cal O}_1$\verb"<>"${\cal O}_2$, ${\cal O}_1$\verb"<"${\cal O}_2$, ${\cal O}_1$\verb">"${\cal O}_2$, ${\cal O}_1$\verb"<="${\cal O}_2$, ${\cal O}_1$\verb">="${\cal O}_2$ \\
\qind{match} & ${\cal T}_1$\verb"~"${\cal T}_2$\\
\qind{negation} & \verb"not" $\cal B$\\
\qind{konjunktion} & ${\cal B}_1$ \verb"and" ${\cal B}_2$\\
\qind{disjunktion} & ${\cal B}_1$ \verb"or" ${\cal B}_2$\\
\qind{skemacheck} & \verb"has("$\cal R$\verb",n)"\\
\qind{typecheck} & \verb"is-Int("$\cal R$\verb",n)", \verb"is-Float("$\cal R$\verb",n)",\\& \verb"is-Bool("$\cal R$\verb",n)", \verb"is-Text("$\cal R$\verb",n)"\\
\qind{maksimum} & \verb"max("$\cal R$\verb",n)"\\
\qind{minimum} & \verb"min("$\cal R$\verb",n)"\\
\qind{funktionsanvendelse} & $\cal F$\verb"("${\cal U}_1$\verb",...,"${\cal U}_k$\verb")"\\
\qind{indskudt udtryk} & \verb"(+ val n"$_1$\verb"="${\cal U}_1$ \ldots
\verb"val n"$_k$\verb"="${\cal U}_k$ \verb"in" $\cal B$ \verb"+)"\\
\qind{betinget udtryk} & \verb"if" $\cal B$ \verb"->" $\cal B$ \ldots \verb"fi"\\
\hline
\end{tabular}
\end{center}
Standardv\ae{}rdien \verb"?-Bool" er en ekstra konstant af typen \verb"Bool", der
blandt andet optr\ae{}der som ``ubestemte'' v\ae{}rdier af \verb"Bool"-attributter.
Der findes ogs\aa{} standardv\ae{}rdier af typerne \verb"Int" og \verb"Text".
Standardv\ae{}rdier omtales n\ae{}rmere i n\ae{}ste kapitel.
\newpage
Udtryk af type \verb"Int" $(\cal I)$:\index{Int}
\begin{center}
\begin{tabular}{|l|l|}
\hline
Betegnelse & Syntaks \\\hline
\qind{konstant} & \ldots, \verb"-2", \verb"-1", \verb"0", \verb"1", 
\verb"2", \verb"3", \ldots \\
\qind{standardv\ae{}rdi} & \verb"?-Int" \\
\qind{navn} & \verb"n" \\
\qind{attribut} & $\cal T$\verb".n" \\
\qind{sum} & ${\cal I}_1$ \verb"+" ${\cal I}_2$\\
\qind{differens} & ${\cal I}_1$ \verb"-" ${\cal I}_2$\\
\qind{produkt} & ${\cal I}_1$ \verb"*" ${\cal I}_2$\\
\qind{kvotient} & ${\cal I}_1$ \verb"/" ${\cal I}_2$\\
\qind{rest} & ${\cal I}_1$ \verb"mod" ${\cal I}_2$\\
\qind{relationsst\o{}rrelse} & \verb"|"$\cal R$\verb"|"\\
\qind{tekstl\ae{}ngde} & \verb"|"$\cal X$\verb"|"\\
\qind{afrunding} & \verb"floor("$\cal D$\verb")", \verb"ceil("$\cal D$\verb")", \verb"round("$\cal D$\verb")"\\
\qind{maksimum} & \verb"max("$\cal R$\verb",n)"\\
\qind{minimum} & \verb"min("$\cal R$\verb",n)"\\
\qind{addition} & \verb"add("$\cal R$\verb",n)"\\
\qind{multiplikation} & \verb"mult("$\cal R$\verb",n)"\\
\qind{opt\ae{}lling} & \verb"count("$\cal R$\verb",n)"\\
\qind{funktionsanvendelse} & $\cal F$\verb"("${\cal U}_1$\verb",...,"${\cal U}_k$\verb")"\\
\qind{indskudt udtryk} & \verb"(+ val n"$_1$\verb"="${\cal U}_1$ \ldots
\verb"val n"$_k$\verb"="${\cal U}_k$ \verb"in" $\cal I$ \verb"+)"\\
\qind{betinget udtryk} & \verb"if" $\cal B$ \verb"->" $\cal I$ \ldots \verb"fi"\\
%\qind{antal dage} & \verb"days("${\cal X}_1$,${\cal X}_2$\verb")"\\
\hline
\end{tabular}
\end{center}
\newpage
Udtryk af type \verb"Float" $(\cal D)$:\index{Int}
\begin{center}
\begin{tabular}{|l|l|}
\hline
Betegnelse & Syntaks \\\hline
\qind{konstant} & f.eks. \verb"-2.0", \verb"1e-10"\\
\qind{standardv\ae{}rdi} & \verb"?-Float" \\
\qind{navn} & \verb"n" \\
\qind{attribut} & $\cal T$\verb".n" \\
\qind{sum} & ${\cal D}_1$ \verb"+" ${\cal D}_2$, ${\cal I}_1$ \verb"+" ${\cal D}_2$ og ${\cal D}_1$ \verb"+" ${\cal I}_2$\\
\qind{differens} & ${\cal D}_1$ \verb"-" ${\cal D}_2$, ${\cal I}_1$ \verb"-" ${\cal D}_2$ og ${\cal D}_1$ \verb"-" ${\cal I}_2$\\
\qind{produkt} & ${\cal D}_1$ \verb"*" ${\cal D}_2$, ${\cal I}_1$ \verb"*" ${\cal D}_2$ og ${\cal D}_1$ \verb"*" ${\cal I}_2$\\
\qind{kvotient} & ${\cal D}_1$ \verb"/" ${\cal D}_2$, ${\cal I}_1$ \verb"/" ${\cal D}_2$ og ${\cal D}_1$ \verb"/" ${\cal I}_2$\\
\qind{rest} & ${\cal I}_1$ \verb"mod" ${\cal D}_2$, ${\cal D}_1$ \verb"mod" ${\cal D}_2$ og ${\cal D}_1$ \verb"mod" ${\cal I}_2$\\
\qind{maksimum} & \verb"max("$\cal R$\verb",n)"\\
\qind{minimum} & \verb"min("$\cal R$\verb",n)"\\
\qind{addition} & \verb"add("$\cal R$\verb",n)"\\
\qind{multiplikation} & \verb"mult("$\cal R$\verb",n)"\\
\qind{funktionsanvendelse} & $\cal F$\verb"("${\cal U}_1$\verb",...,"${\cal U}_k$\verb")"\\
\qind{indskudt udtryk} & \verb"(+ val n"$_1$\verb"="${\cal U}_1$ \ldots
\verb"val n"$_k$\verb"="${\cal U}_k$ \verb"in" $\cal I$ \verb"+)"\\
\qind{betinget udtryk} & \verb"if" $\cal B$ \verb"->" $\cal I$ \ldots \verb"fi"\\
\qind{trigonometri} & \verb"sin("$\cal D$\verb")", \verb"cos("$\cal D$\verb")", \verb"tan("$\cal D$\verb")",\\
&\verb"asin("$\cal D$\verb")", \verb"acos("$\cal D$\verb")", \verb"atan("$\cal D$\verb")", \verb"atan2("$\cal D$, $\cal D$\verb")"\\
\qind{eksponent} & \verb"sqrt("$\cal D$\verb")", \verb"pow("$\cal D$, $\cal D$\verb")", \verb"abs("$\cal D$\verb")" \\
%\qind{antal dage} & \verb"days("${\cal X}_1$,${\cal X}_2$\verb")"\\
\hline
\end{tabular}
\end{center}
\newpage
Udtryk af type \verb"Text" $(\cal X)$:\index{Text}
\begin{center}
\begin{tabular}{|l|l|}
\hline
Betegnelse & Syntaks \\\hline
\qind{konstant} & \verb$""$, \verb$"AGF"$, \ldots \\
\qind{standardv\ae{}rdi} & \verb"?-Text" \\
\qind{navn} & \verb"n" \\
\qind{attribut} & $\cal T$\verb".n" \\
\qind{deltekst} & \verb"substr("$\cal X$\verb", "${\cal I}_1$\verb", "${\cal I}_2$\verb")"\\
\qind{konkatenation} & ${\cal X}_1$ \verb"++" ${\cal X}_2$\\
\qind{maksimum} & \verb"max("$\cal R$\verb",n)"\\
\qind{minimum} & \verb"min("$\cal R$\verb",n)"\\
\qind{funktionsanvendelse} & $\cal F$\verb"("${\cal U}_1$\verb",...,"${\cal U}_k$\verb")"\\
\qind{indskudt udtryk} & \verb"(+ val n"$_1$\verb"="${\cal U}_1$ \ldots
\verb"val n"$_k$\verb"="${\cal U}_k$ \verb"in" $\cal X$ \verb"+)"\\
\qind{betinget udtryk} & \verb"if" $\cal B$ \verb"->" $\cal X$ \ldots \verb"fi"\\
\qind{pr\ae{}fiks} & \verb"before("${\cal X}_1$,${\cal X}_2$\verb")"\\
\qind{suffiks} & \verb"after("${\cal X}_1$,${\cal X}_2$\verb")"\\
%\qind{dags dato} & \verb"today"\\
%\qind{fremtidig dato} & \verb"date("${\cal X}$,${\cal I}$\verb")"\\
\hline
\end{tabular}
\end{center}
\newpage
Udtryk af type \verb"Tup" $(\cal T)$:\index{Tup}
\begin{center}
\begin{tabular}{|l|l|}
\hline
Betegnelse & Syntaks \\\hline
\qind{konstant} & \verb"Tup(n"$_1$\verb":"${\cal A}_1$\verb",...,n"$_k$\verb":"${\cal A}_k$\verb")"  \\
\qind{navn} & \verb"n", \verb"#"\\
\qind{opdatering} & ${\cal T}_1$ \verb"<<" ${\cal T}_2$\\
\qind{elimination} & $\cal T$ $\backslash$ \verb"n"\\
\qind{funktionsanvendelse} & $\cal F$\verb"("${\cal U}_1$\verb",...,"${\cal U}_k$\verb")"\\
\qind{indskudt udtryk} & \verb"(+ val n"$_1$\verb"="${\cal U}_1$ \ldots
\verb"val n"$_k$\verb"="${\cal U}_k$ \verb"in" $\cal T$ \verb"+)"\\
\qind{betinget udtryk} & \verb"if" $\cal B$ \verb"->" $\cal T$ \ldots \verb"fi"\\
\hline
\end{tabular}
\end{center}

Udtryk af type \verb"Rel" $(\cal R)$:\index{Rel}
\begin{center}
\begin{tabular}{|l|l|}
\hline
Betegnelse & Syntaks \\\hline
\qind{konstant} & \verb"rel("$\cal T$\verb")", \verb"zero"\\
\qind{navn} & \verb"n"\\
\qind{union} & ${\cal R}_1$ \verb"+" ${\cal R}_2$\\
\qind{join} & ${\cal R}_1$ \verb"*" ${\cal R}_2$\\
\qind{difference} & ${\cal R}_1$ \verb"-" ${\cal R}_2$\\
\qind{project} & $\cal R$ \verb"|+" \verb"n"$_1$\verb",...,n"$_k$, $\cal R$ \verb"|-"
\verb"n"$_1$\verb",...,n"$_k$\\
\qind{select} & $\cal R$ \verb"?" $\cal B$\\
\qind{rename} & $\cal R$ \verb"[n"$_1$\verb" <- n"$_2$\verb"]"\\
\qind{forall} & \verb"!("${\cal R}_1$\verb"): "${\cal R}_2$\\
\qind{factor} & \verb"!("${\cal R}_1,\ldots,{\cal R}_m$\verb")|"\verb"n"$_1$\verb",...,n"$_k$: ${\cal R}$\\
\qind{funktionsanvendelse} & $\cal F$\verb"("${\cal U}_1$\verb",...,"${\cal U}_k$\verb")"\\
\qind{indskudt udtryk} & \verb"(+ val n"$_1$\verb"="${\cal U}_1$ \ldots
\verb"val n"$_k$\verb"="${\cal U}_k$ \verb"in" $\cal R$ \verb"+)"\\
\qind{betinget udtryk} & \verb"if" $\cal B$ \verb"->" $\cal R$ \ldots \verb"fi"\\
\hline
\end{tabular}
\end{center}

Udtryk af type \verb"Func" $(\cal F)$:\index{Func}
\begin{center}
\begin{tabular}{|l|l|}
\hline
Betegnelse & Syntaks \\\hline
\qind{konstant} & \verb"Func (...) -> (...)" $\cal U$ \verb"end"\\
\qind{navn} & \verb"n"\\
\qind{funktionsanvendelse} & $\cal F$\verb"("${\cal U}_1$\verb",...,"${\cal U}_k$\verb")"\\
\qind{indskudt udtryk} & \verb"(+ val n"$_1$\verb"="${\cal U}_1$ \ldots
\verb"val n"$_k$\verb"="${\cal U}_k$ \verb"in" $\cal F$ \verb"+)"\\
\qind{betinget udtryk} & \verb"if" $\cal B$ \verb"->" $\cal F$ \ldots \verb"fi"\\
\hline
\end{tabular}
\end{center}

\newpage
\section{Overordnede begreber}
Efter de foreg\aa{}ende kapitlers eksempelbaserede pr\ae{}sentation
skal vi nu give en mere systematisk og begrebsorienteret beskrivelse af
{\sc Rasmus}.

\subsection{Relationelle databaser}
{\sc Rasmus} er interessant af mindst to \aa{}rsager, dels fordi
sproget indeholder de v\ae{}sentligste ingredienser fra det emneomr\aa{}de,
der hedder {\em \qind{relationelle databaser}}, og dels fordi det repr\ae{}senterer
nogle af de mest fundamentale begreber, der findes i programmeringssprog
og programmeringsnotationer.

Relationelle databaser blev ``opfundet'' omkring 1970, som et forslag
til en ``ren'', systematisk og teoretisk velfunderet teknik til
opbevaring og behandling af store datam\ae{}ngder. Den basale
iagttagelse var, at dels kan praktisk talt enhver samling informationer,
man kan forestille sig, repr\ae{}senteres i form af en eller flere
relationer, og dels tillader de basale relationelle operatorer
(union, join, select, project), at man kombinerer og udtr\ae{}kker
informationer fra vilk\aa{}rlige relationer p\aa{} en meget generel
m\aa{}de. 

Dette betyder blandt andet, at hvis information opbevares
i form af relationer, s\aa{} kan forskellige data, der fra f\o{}dslen
ikke synes at have meget med hinanden at g\o{}re, bearbejdes af det samme
{\em databehandlingssystem}, uanset tidspunktet for og omst\ae{}ndighederne
ved det tilblivelse. Dette giver en fleksibilitet og en generalitet,
der st\aa{}r i skarp kontrast til en strategi, hvor man designer b\aa{}de
datarepr\ae{}sentation og tilh\o{}rende programmer afh\ae{}ngigt af
den konkrete opgave og uden at tage h\o{}jde for, at det senere skal kunne
``samk\o{}res'' med andre data. Relationelle databaser introducerer derfor
ikke blot nogle nyttige begreber og en hensigtsm\ae{}ssig systematik,
teknikken sparer ogs\aa{} ressourcer i forbindelse med praktisk 
databehandling.

\subsection{Typer, v\ae{}rdier og udtryk}
Uanset den praktiske nyttev\ae{}rdi, som bestemt er til stede, skal vi
i det f\o{}lgende koncentere os om det andet interessante aspekt, nemlig
{\sc Rasmus} som programmeringssprog. 

For at forst\aa{} et 
programmeringssprog, herunder {\sc Rasmus}, er det 
afg\o{}rende at forst\aa{}
det centrale begreb {\em v\ae{}rdi}, som er en temmelig abstrakt
st\o{}rrelse. St\ae{}rkt forenklet kan man sige, at faciliteterne i
{\sc Rasmus} er beregnet til at {\em skabe}, {\em gemme\/}
og {\em genfinde\/} v\ae{}rdier: heltal, sandhedsv\ae{}rdier,
tegnf\o{}lger, tupler, relationer og funktioner.
V\ae{}rdierne skabes som resultater af udtryk, hvorfor udtryk ogs\aa{}
klassificeres efter hvilken slags v\ae{}rdier, de producerer.
I datalogisk terminologi bruger man {\em typer}, det vil sige,\index{type}
at vi i {\sc Rasmus} har v\ae{}rdier og udtryk af typerne:
\verb"Int", \verb"Float", \verb"Bool", \verb"Text", \verb"Tup", \verb"Rel" og \verb"Func".

\subsection{Konstante udtryk}
Ethvert udtryk skaber (eller producerer) alts\aa{} en v\ae{}rdi af den
p\aa{}g\ae{}ldende type. De simpleste udtryk er {\em konstanterne},
som fx\index{konstant}
$$\mbox{\tt 5}$$
Dette er et udtryk af type \verb"Int", der som resultat (eller v\ae{}rdi)
har {\em tallet\/} 5. Bem\ae{}rk, at {\em tallet\/} 5 i sig selv er noget
abstrakt og at udtrykket {\tt 5} blot er et af mange mulige udtryk, hvis
v\ae{}rdi er dette tal. Hvis man fx var forelsket i romertal eller
bin\ae{}re tal, s\aa{} kunne man bruge \verb"V" eller \verb"101" som
udtryk med v\ae{}rdi 5. I {\sc Rasmus} benyttes den velkendte decimale
notation for konstanter af type \verb"Int".

De \o{}vrige typer har ogs\aa{} konstanter, det vil sige udtryk,
der angiver deres v\ae{}rdier. Som n\ae{}vnt i kapitel 4 er
f\o{}lgende eksempler karakteristiske for konstante udtryk af de
forskellige typer
{\tt
  \begin{description}%% {{\tt ~~~Text}:}
  \item[{\tt ~~~Float}:] 1.2
\item[{\tt ~~~Bool}:] true 
\item[{\tt ~~~Text}:] "AGF"
\item[{\tt ~~~Tup}:] tup(hold:"AGF")
\item[{\tt ~~~Rel}:] rel(tup(hold:"AGF"))
\item[{\tt ~~~Func}:] func () -> (Text) "AGF" end
\end{description}
}
Deres v\ae{}rdier er henholdsvis
\begin{itemize}
\item decimal tallet 1.2
\item sandhedsv\ae{}rdien sand
\item tegnf\o{}lgen best\aa{}ende af tegnene \verb"A", \verb"G" og \verb"F"
\item et tupel med attribut \verb"hold" hvis tilsvar\-en\-de v\ae{}rdi er
tegnf\o{}lgen best\aa{}\-en\-de af tegnene \verb"A", \verb"G" og \verb"F"
\item en relation indeholdende et enkelt tupel med attribut \verb"hold" og
til\-svar\-en\-de v\ae{}rdi
tegnf\o{}lgen best\aa{}ende af tegnene \verb"A", \verb"G" og \verb"F"
\item en funktion uden parametre, der returnerer tegnf\o{}lgen best\aa{}\-en\-de 
af tegnene \verb"A", \verb"G" og \verb"F"
\end{itemize}
Enhver type har en {\em m\ae{}ngde\/} af v\ae{}rdier knyttet til sig.
V\ae{}rdim\ae{}ngderne i {\sc Rasmus} er\index{v\ae{}rdim\ae{}ngde}
\begin{description}%% {{\tt ~~~Text}:}
\item[{\tt ~~~Int}:] m\ae{}ngden af heltal
\item[{\tt ~~~Float}:] m\ae{}ngden af komma tal (IEEE 754, double)
\item[{\tt ~~~Bool}:] de to sandhedsv\ae{}rdier
\item[{\tt ~~~Text}:] m\ae{}ngden af f\o{}lger af ASCII tegn
\item[{\tt ~~~Tup}:] m\ae{}ngden af tupler
\item[{\tt ~~~Rel}:] m\ae{}ngden af relationer, der hver er en endelig
m\ae{}ngde af tupler med samme skema
\item[{\tt ~~~Func}:] m\ae{}ngden af funktioner med nul eller flere
argumenter af en af de seks typer og med et resultat, der er af en af de seks
typer
\end{description}
V\ae{}rdim\ae{}ngden i \verb"Int", \verb"Float", \verb"Bool", \verb"Text", \verb"Tup" og
\verb"Rel" er {\em ordnede}, det vil sige, man kan sammenligne deres
v\ae{}rdier. Ordningerne er som f\o{}lger
\begin{description}%% {{\tt ~~~Text}:}
\item[{\tt ~~~Int}:] den s\ae{}dvanlige ordning p\aa{} heltallene
\item[{\tt ~~~Float}:] den s\ae{}dvanlige ordning p\aa{} decimaltallene
\item[{\tt ~~~Bool}:] \verb"false" er mindre end \verb"true"
\item[{\tt ~~~Text}:] alfabetisk ordning, hvor ``bogstaverne'' er ASCII tegn
i den s\ae{}dvanlige r\ae{}kkef\o{}lge
\item[{\tt ~~~Tup}:] {\tt t$_1$} er mindre end {\tt t$_2$}, hvis alle attributnavne
i {\tt t$_1$} ogs\aa{} forekommer i {\tt t$_2$} og v\ae{}rdier h\o{}rende
til f\ae{}lles attributnavne er ens i de to tupler. 
\item[{\tt ~~~Rel}:] {\tt r$_1$} er mindre end {\tt r$_2$}, hvis 
de har samme skema, og alle tupler i {\tt r$_1$} ogs\aa{} 
forekommer i {\tt r$_2$}
\end{description}
For de tre atomare typer (\verb"Int", \verb"Float", \verb"Bool" og \verb"Text")
er dette imidlertid ikke hele sandheden. De fire typer, hvis
v\ae{}rdier er de eneste mulige attributv\ae{}rdier, har endnu en v\ae{}rdi,
nemlig den s\aa{}kaldte {\em \qind{standardv\ae{}rdi}}. Disse 
standardv\ae{}rdier kan angives
med konstantudtrykkene \verb"?-Int", \verb"?-Float", \verb"?-Bool" og \verb"?-Text".
Meningen med og behovet for s\aa{}danne standardv\ae{}rdier kan
illustreres af f\o{}lgende udvidelse af informationen om vore
fodboldkampe. Relationen
\verb"Tips" indeholder kun kampe fra \verb"Runde1", den f\o{}rste halvdel af 
turneringen. Dette skyldes, at spilledagene i anden halvdel falder
p\aa{} en s\aa{}dan m\aa{}de, at de seks tophold ikke spiller
inden for samme uge p\aa{} en s\aa{}dan m\aa{}de, at vi kan
definere en tipskupon med tre kampe, hvor alle hold deltager.
Hvis man nu tillader andre kampe at optr\ae{}de p\aa{} kuponen, er det
naturligvis nemt ogs\aa{} at lave tipskuponer for \verb"Runde2".
I s\aa{} falde kunne \verb"Tips" relationen have f\o{}lgende udseende
\begin{center}
\RERRR{Id:Int}{Uge:Int}{Nr:Int}{
$\vdots$ & 
$\vdots$ & 
$\vdots$ \\
12 &
18 &
3\\
13 &
19 &
1\\
14 &
19 &
2\\
15 &
19 &
3\\
19 & 21 & 1\\
20 & 21 & 2\\
\grey\grey\grey\grey\grey\grey\grey\grey\grey\grey\grey\grey & 21 & 3\\
\grey\grey\grey\grey\grey\grey\grey\grey\grey\grey\grey\grey & 23 & 1\\
26 & 23 & 2\\
27 & 23 & 3\\
29 & 25 & 1\\
30 & 25 & 1\\
\grey\grey\grey\grey\grey\grey\grey\grey\grey\grey\grey\grey & 25 & 3}
\end{center}
Som det ses, er der nogle af kampene for hvilke \verb"Id"-feltet
er ``tomt''. Grunden er, at vi kun har Superliga kampe i vores
database, det vil sige, vi mangler identifikationen af den relevante
kamp. Nu kan felter i tupler imidlertid ikke v\ae{}re tomme --
{\sc Rasmus} forlanger, at der altid er knyttet en v\ae{}rdi til
en attribut -- og derfor skal vi bruge en v\ae{}rdi, der angiver
``manglende information''. Dette er pr\ae{}cis hvad standardv\ae{}rdien
kan bruges til. De ``tomme felter'' er alts\aa{} i virkeligheden ikke
tomme; de indeholder \verb"?-Int", standardv\ae{}rdien for typen \verb"Int".
I {\sc Rasmus} systemet angives dette ved, at feltet er {\em lysegr\aa{}t}.
Vi kan illustrere sammenh\ae{}ngen mellem standardv\ae{}rdien
\verb"?-Int" og et lysegr\aa{}t felt ved hj\ae{}lp af f\o{}lgende eksempel.

\eksempel{
V\ae{}rdien af udtrykket 
\begin{center}
\verb"rel(tup(Hovsa:?-Int))"
\end{center}
er
\begin{center}
\RER{Hovsa:Int}{\grey\grey\grey\grey\grey\grey\grey\grey\grey\grey\grey\grey\grey\grey\grey\grey\grey\grey}
\end{center}
}

Udover at angive manglende information kan standardv\ae{}rdien
ogs\aa{} brug\-es til at angive at et felt er irrelevant for et
givent tupel, eller at der er tale om andre former for undtagelser.
Vi bem\ae{}rker, at et tom {\em hvidt\/} felt ikke angiver en
standardv\ae{}rdi men derimod en \verb"Text"-attribut, hvis v\ae{}rdi
er den tomme tegnf\o{}lge.

\subsection{Navne og tilstande}
Vi vender nu tilbage til beskrivelsen af udtryk, hvor vi allerede har
set, at konstanter er udtryk. En anden form for simple udtryk er navne,
som fx
$$ \mbox{\tt Kampe} $$
V\ae{}rdien af et s\aa{}dant udtryk er den v\ae{}rdi, der er {\em bundet\/}
til det p\aa{}\-g\ae{}ld\-en\-de navn. I eksemplet er det den relation,
der indeholder samtlige kampe i turneringen. N\aa{}r man introducerer et
navn i {\sc Rasmus} er det altid for at navngive en v\ae{}rdi af en af
de seks typer, det vil sige, alle navne i {\sc rasmus} har altid
en v\ae{}rdi bundet til sig. Den samling af navne, med tilh\o{}rende
v\ae{}rdier, man p\aa{} et givet tidspunkt har adgang til, kaldes
under et for den aktuelle {\em \qind{tilstand}}, og udregningen af udtryk i
{\sc Rasmus} foreg\aa{}r altid relativt til en s\aa{}dan tilstand.

Der findes tre m\aa{}der, hvorp\aa{} man kan binde v\ae{}rdier til navne
og derved modificere den aktuelle tilstand.
Den f\o{}rste m\aa{}de er ved at navngive et udtryk (ved hj\ae{}lp af
{\bf Save last}) hvilket i eksemplerne er angivet med \fbox{\fbox{\tt Navn}}.
Effekten heraf er, at den p\aa{}g\ae{}ld\-en\-de v\ae{}rdi bliver bundet
til \verb"Navn", hvorefter denne binding tilf\o{}jes til tilstanden.
Hvis der allerede findes en binding til \verb"Navn", s\aa{} forsvinder
denne. Vi kalder en s\aa{}dan udvidelse/overskrivning for en {\em \qind{opdatering}\/}
af tilstanden.

Den anden m\aa{}de at introducere bindinger er ved hj\ae{}lp af et
indskudt udtryk af formen
{\tt
\begin{center}
\begin{tabular}{ll}
(+ & val $\META{id}_{\mbox{\bf 1}}$ = $\META{udtryk}_{\mbox{\bf 1}}$ \\
   & val $\META{id}_{\mbox{\bf 2}}$ = $\META{udtryk}_{\mbox{\bf 2}}$ \\
   & $\vdots$ \\
   & val $\META{id}_{\mbox{\bf n}}$ = $\META{udtryk}_{\mbox{\bf n}}$ \\
   & in \META{udtryk}\\
+) &
\end{tabular}
\end{center}
}
Her sker der f\o{}lgende. F\o{}rst udregnes $\META{udtryk}_{\mbox{\bf 1}}$
i den aktuelle tilstand og dets v\ae{}rdi bindes til
$\META{id}_{\mbox{\bf 1}}$, hvorefter den aktuelle tilstand opdateres
med denne binding. Derefter udregnes $\META{udtryk}_{\mbox{\bf 2}}$
i den nye tilstand, v\ae{}rdien bindes til $\META{id}_{\mbox{\bf 2}}$
og tilstanden opdateres med denne binding. Dette fors\ae{}tter indtil
alle {\bf n} opdateringer er foretaget, hvorefter \META{udtryk}
udregnes i den resulterende tilstand. Herefter {\em retableres\/}
den oprindelige tilstand, det vil sige, udregningen af et indskudt
udtryk p\aa{}virker s\aa{}ledes ikke tilstanden -- de udf\o{}rte
opdateringer er {\em lokale\/} og {\em midlertidige}.

Den sidste m\aa{}de at foretage bindinger p\aa{} er i forbindelse med
funktionsanvendelse. Hvis funktionen
\begin{center}
\begin{tabular}{l}
\verb"func (p1:T1, p2:T2) -> (T)"\\
\SS{2}{\bf udtryk}\\
\verb"end"
\end{tabular}
\end{center}
er bundet til navnet \verb"F" i den aktuelle tilstand, vil
funktionsanvendelsen
\begin{center}
\verb"F("\META{udtryk}$_{\mbox{\bf 1}}$\verb","\META{udtryk}$_{\mbox{\bf 2}}$\verb")"
\end{center}
bevirke f\o{}lgende. F\o{}rst udregnes v\ae{}rdierne af 
\META{udtryk}$_{\mbox{\bf 1}}$ og \META{udtryk}$_{\mbox{\bf 2}}$
i den aktuelle tilstand. Dern\ae{}st bindes disse v\ae{}rdier til
henholdsvis \verb"p1" og \verb"p2", og tilstanden opdateres med disse
to bindinger. Nu udregnes {\bf udtryk} i den s\aa{}ledes
modificerede tilstand, og den resulterende v\ae{}rdi er ogs\aa{}
v\ae{}rdien af funktionsanvendelsen. Endelig retableres den oprindelige
tilstand. Udregningen af en funktionsanvendelse p\aa{}virker alts\aa{}
heller ikke tilstanden.

\subsection{Sammensatte udtryk}
Efter at have introduceret simple udtryk (konstanter og navne)
og tilstande, skal vi nu se p\aa{} {\em sammensatte} udtryk,
der f\aa{}s ved p\aa{} s\ae{}dvanlig algebraisk vis at
sammens\ae{}tte to deludtryk ved hj\ae{}lp af en {\em operator}.
Et simpelt eksempel er
$$ \mbox{\tt 3+4} $$
hvor operatoren \verb"+" er heltalsaddition og de to
deludtryk er konstanter af type \verb"Int". Det er klart, at \verb"+"
stiller det krav, at operanderne skal v\ae{}re heltal. Fx er udtrykket
$$ \mbox{\verb:3+"AGF":} $$
hverken tilladt eller s\ae{}rligt meningsfuldt, idet det naturligvis
ikke giver mening at addere tallet \verb"3" og
tegnf\o{}lgen \verb"A", \verb"G", \verb"F".
Om et udtryk som
$$ \mbox{\verb:3+x:} $$
er tilladt, afh\ae{}nger af typen af den v\ae{}rdi, der er bundet til
\verb"x". Hvis den er af type \verb"Int" er sagen i orden, ellers er
det en fejl.

Vi gennemg\aa{}r nu udtrykkene i oversigten i afsnit 5.8 og kommenterer
i n\o{}dvendigt omfang operatorernes betydning, samt hvilke krav de
stiller til deres operander.

{\bf For samtlige seks udtrykstyper g\ae{}lder f\o{}lgende.}

{\em Konstanter:}\index{konstant}\\
Taler for sig selv. Resultatet er den ``oplagte'' (abstrakte) v\ae{}rdi.

{\em Navne:}\index{navn}\\
Et navn skal findes i tilstanden, og v\ae{}rdien er den,
der her er bundet til navnet.

{\em Funktionsanvendelse $\cal F$\verb"("${\cal U}_1$\verb",...,"${\cal U}_k$
\verb")":}\index{funktionsanvendelse}\\
${\cal U}_1$, \ldots, ${\cal U}_k$ skal v\ae{}re udtryk af samme typer
som de tilsvarende formelle parametre i den funktion, der er
v\ae{}rdien af $\cal F$. Resultatet er som beskrev\-et i afsnit 6.4.

{\em Indskudt udtryk \verb"(+ val" \ldots \verb"in" $\cal U$ \verb"+)":}\\
Som beskrevet i afsnit 6.4.\index{indskudt udtryk}

{\em Betinget udtryk \verb"if" ${\cal B}_1$ \verb"->" ${\cal U}_1$
\verb"&" \ldots \verb"&" ${\cal B}_n$ \verb"->" ${\cal U}_n$ \verb"fi":}\index{betinget udtryk}\\
Resultatet opn\aa{}s ved f\o{}rst at udregne ${\cal B}_1$, \ldots,
${\cal B}_n$ i r\ae{}kkef\o{}lge indtil det f\o{}rste ${\cal B}_i$,
hvis v\ae{}rdi er sand. Resultatet er da v\ae{}rdien af
${\cal U}_i$. Hvis intet ${\cal B}_i$ er sandt, s\aa{} er resultatet
(lidt arbitr\ae{}rt) tallet 0.

{\bf For de {\em atomare typer\/} g\ae{}lder f\o{}lgende.}\index{atomar}

{\em Standardv\ae{}rdien \verb"?-"$\cal A$:}\index{standardv\ae{}rdi}\\
Dette er en ``ekstra'' v\ae{}rdi, der tilf\o{}jes til typen.
Den er kun lig med sig selv og st\aa{}r ikke i nogen ordningsrelation
til de andre v\ae{}rdier. Standardv\ae{}rdien m\aa{} kun v\ae{}re
operand til sammenligningsoperatorer.

{\em Attribut $\cal T$\verb".n":}\index{attribut}\\
Navnet \verb"n" skal v\ae{}re en attribut i v\ae{}rdien af $\cal T$.
Resultatet er den til\-svar\-en\-de attributv\ae{}rdi.

{\em Maksimum \verb"max("$\cal R$\verb",n)":}\index{maksimum}\\
Navnet \verb"n" skal v\ae{}re en attribut i v\ae{}rdien af $\cal R$.
Resultatet er den st\o{}rste \verb"n"-v\ae{}rdi i $\cal R$ i den
p\aa{}g\ae{}ldende types ordning. Hvis $\cal R$ er tom, eller
\verb"n"-attributten overalt har standardv\ae{}rdien \verb"?-"$\cal A$,
er resultatet denne standardv\ae{}rdi.

{\em Minimum \verb"min("$\cal R$\verb",n)":}\index{minimum}\\
Analog til maksimum.

{\bf For \verb"Bool"-udtryk g\ae{}lder f\o{}lgende.}

{\em Sammenligning ${\cal O}_1$\verb"="${\cal O}_2$, ${\cal O}_1$\verb"<"${\cal O}_2$, og s\aa{} videre:}\index{sammenligning}\\
${\cal O}_1$ og ${\cal O}_2$ skal v\ae{}re af samme type. Resultatet er
\verb"true", hvis v\ae{}rdierne har den angivne indbyrdes ordning, og
\verb"false" ellers.

{\em Negation \verb"not" $\cal B$:}\index{negation}\\
Resultatet er \verb"true" (\verb"false") hvis v\ae{}rdien af 
$\cal B$ er \verb"false" (\verb"true").

{\em Konjunktion ${\cal B}_1$ \verb"and" ${\cal B}_2$:}\index{konjunktion}\\
Resultatet er \verb"true", hvis b\aa{}de ${\cal B}_1$ og ${\cal B}_2$
har v\ae{}rdi \verb"true", og \verb"false" ellers.

{\em Disjunktion ${\cal B}_1$ \verb"or" ${\cal B}_2$:}\index{disjunktion}\\
Resultatet er \verb"false", hvis b\aa{}de ${\cal B}_1$ og ${\cal B}_2$
har v\ae{}rdi \verb"false", og \verb"true" ellers.

{\em Skemacheck \verb"has("$\cal R$\verb",n)":}\index{skemacheck}\\
Resultatet er \verb"true", hvis \verb"n" er attributnavn i $\cal R$,
og \verb"false" ellers.

{\em Typecheck \verb"is-"$\cal A$\verb"("$\cal R$\verb",n)":}\index{typecheck}\\
Resultatet er \verb"true", hvis \verb"n"'s type i $\cal R$'s type
er $\cal A$, og \verb"false" ellers.

{\em Match ${\cal X}_1$\verb"~"${\cal X}_2$:}\index{match}\\
Resultatet er \verb"true", hvis ${\cal X}_1$ optr\ae{}der som en del af ${\cal
X}_2$.

{\bf For \verb"Int"-udtryk g\ae{}lder f\o{}lgende.}

{\em Regneoperatorer \verb"+", \verb"-", \verb"*", \verb"/", \verb"mod":}\\
Taler for sig selv.\index{regneoperator}

{\em Relationsst\o{}rrelse og tekstl\ae{}ngde, \verb"|"$\cal R$\verb"|" og
\verb"|"$\cal X$\verb"|":}\index{relationsst\o{}rrelse}\index{tekstl\ae{}ngde}\\
Resultatet er antallet af tupler i $\cal R$ og antallet af tegn i
$\cal X$.

{\em Addition \verb"add("$\cal R$\verb",n)":}\index{add}\\
Navnet \verb"n" skal v\ae{}re en attribut i $\cal R$ af type \verb"Int".
Resultatet er summen af \verb"n"-v\ae{}rdierne i $\cal R$. Hvis 
$\cal R$ er tom, eller \verb"n"-v\ae{}rdierne alle er \verb"?-Int", er
resultatet \verb"?-Int".

{\em Multiplikation \verb"mult("$\cal R$\verb",n)":}\index{mult}\\
Analog til addition.

{\em Opt\ae{}lling \verb"count("$\cal R$\verb",n)":}\index{count}\\
Navnet \verb"n" skal v\ae{}re en attribut i $\cal R$.
Resultatet er antallet af \verb"n"-v\ae{}rdierne i $\cal R$, der er
forskellige fra standardv\ae{}rdien.

{\em Antal dage \verb"days("${\cal T}_1$,${\cal T}_2$\verb")":}\index{days}\index{antal dage}\\
Resultatet er antallet af dage mellem datoerne ${\cal T}_1$ og ${\cal T}_2$.
En dato repr\ae{}senteres som en tekst p\aa{} formen {\tt dd/mm/\aa{}\aa{}}.

{\bf For \verb"Text"-udtryk g\ae{}lder f\o{}lgende.}

{\em Deltekst \verb"substr("$\cal X$\verb", "${\cal I}_1$\verb","${\cal I}_2$\verb")":}\\
Resultatet er delteksten af $\cal X$ best\aa{}ende af de tegn, hvis
indices er st\o{}rre end eller lig med v\ae{}rdien af 
${\cal I}_1$ om mindre end v\ae{}rdien af ${\cal I}_2$.\index{deltekst}

{\em Konkatenation ${\cal X}_1$ \verb"++" ${\cal X}_2$:}\index{konkatenation}\\
Resultatet er sammens\ae{}tningen af v\ae{}rdierne af
${\cal X}_1$ og ${\cal X}_2$.

{\em Pr\ae{}fiks \verb"before("${\cal X}_1$,${\cal X}_2$\verb")":}\index{before}\index{pr\ae{}fiks}\\
Resultatet er den del af ${\cal X}_2$, der st\aa{}r foran f\o{}rste forekomst af
delteksten ${\cal X}_1$. Hvis ${\cal X}_1$ slet ikke forekommer, s\aa{} er
resultatet den tomme tekst.

{\em Suffiks \verb"after("${\cal X}_1$,${\cal X}_2$\verb")":}\index{after}\index{suffiks}\\
Resultatet er den del af ${\cal X}_2$, der st\aa{}r efter f\o{}rste forekomst af
delteksten ${\cal X}_1$. Hvis ${\cal X}_1$ slet ikke forekommer, s\aa{} er
resultatet den tomme tekst.

%% {\em Dags dato \verb"today":}\index{dags dato}\index{today}\\
%% Resultatet er dags dato p\aa{} formen {\tt dd/mm/\aa{}\aa{}}.

%% {\em Fremtidig dato \verb"date("${\cal X}$,${\cal I}$\verb")":}\index{fremtidig dato}\index{date}\\
%% Resultatet er den dato, der ligger ${\cal I}$ dage senere end datoen ${\cal X}$.
%% En dato repr\ae{}senteres som en tekst p\aa{} formen {\tt dd/mm/\aa{}\aa{}}.

{\bf For \verb"Tup"-udtryk g\ae{}lder f\o{}lgende.}

{\em Opdatering ${\cal T}_1$ \verb"<<" ${\cal T}_2$:}\index{opdatering}\\
Resultatet er v\ae{}rdien af ${\cal T}_1$ opdateret med
bindingerne i v\ae{}rdien af ${\cal T}_2$.

{\em Elimination $\cal T$ $\backslash$ \verb"n":}\index{elimination}\\
Resultatet er v\ae{}rdien af $\cal T$ med en eventuel binding
til \verb"n" fjernet.

{\bf For \verb"Rel"-udtryk g\ae{}lder f\o{}lgende.}

{\em Konstanten \verb"zero":}\index{zero}\\
Resultatet er den tomme relation med tomt skema.

{\em Union og difference, ${\cal R}_1$ \verb"+" ${\cal R}_2$ og
${\cal R}_1$ \verb"-" ${\cal R}_2$:}\index{union}\index{difference}\\
${\cal R}_1$ og ${\cal R}_2$ skal have samme skema. Resultatet er
henholdsvis foreningsm\ae{}ngde og m\ae{}ngdedifferens af tuplerne
i ${\cal R}_1$ og ${\cal R}_2$.

{\em Project $\cal R$ \verb"|+" \verb"n"$_1$\verb",...,n"$_k$}\index{project}
Navnene \verb"n"$_1$, \ldots, \verb"n"$_k$ skal v\ae{}re attributter
i $\cal R$. Resultatet er m\ae{}ngden af tupler i $\cal R$ begr\ae{}nset
til disse attributter.

{\em Select $\cal R$ \verb"?" $\cal B$:}\index{select}\\
Resultatet er den delm\ae{}ngde af tupler i $\cal R$, hvor
$\cal B$ har v\ae{}rdi \verb"true", n\aa{}r symbolet \verb"#" erstattes
med det p\aa{}g\ae{}ldende tupel.

{\em Rename $\cal R$ \verb"[n"$_1$\verb" <- n"$_2$\verb"]":}\index{rename}\\
Navnet \verb"n"$_1$ skal v\ae{}re en attribut i $\cal R$, hvorimod
navnet \verb"n"$_2$ ikke m\aa{} v\ae{}re det. Resultatet er
m\ae{}ngden af tupler i $\cal R$ med \verb"n"$_1$ omd\o{}bt
til \verb"n"$_2$.

{\em Join ${\cal R}_1$ \verb"*" ${\cal R}_2$:}\index{join}\\
F\ae{}lles attributter i ${\cal R}_1$ og ${\cal R}_2$ skal have samme type.
Resultatets skema er foreningen af de to skemaer. 
Resultatets tupler er alle de, der projiceret p\aa{}
${\cal R}_1$'s skema tilh\o{}rer ${\cal R}_1$, og projiceret p\aa{}
${\cal R}_2$'s skema tilh\o{}rer ${\cal R}_2$.

{\em Forall \verb"!("${\cal R}_1$\verb"): "${\cal R}_2$:}\index{forall}\\
Resultatet er foreningsm\ae{}ngden af de relationer, 
der fremkommer ved for hvert
${\cal R}_1$-tupel at  udregne ${\cal R}_2$ med symbolet
\verb"#" erstattet med tuplet.

{\em Factor \verb"!("${\cal R}_1,\ldots,{\cal R}_m$\verb")|"\verb"n"$_1$\verb",...,n"$_k$\verb":"${\cal R}$:}\index{factor}\\
Resultatet er foreningsm\ae{}ngden af de relationer,
der fremkommer ved for hvert faktorsekvens $(t_i,r_{i1},\ldots,r_{im})$ at
beregne ${\cal R}$ med symbolet
\verb"#" erstattet med $t_i$ og symbolet \verb"@(j)" erstattet med $r_{ij}$.

{\bf For \verb"Func"-udtryk g\ae{}lder f\o{}lgende.}

{\em Konstant \verb"func (...) -> (...)" $\cal U$ \verb"end":}\index{konstant}\\
Typen af $\cal U$ skal v\ae{}re resultattypen.

\subsection{Algebraiske love}
Som det fremg\aa{}r af det foreg\aa{}ende, spiller operatorerne
den helt centrale rolle i {\sc Rasmus}. Vi skal derfor i resten af
dette kapitel interessere os lidt mere for deres {\em algebraiske\/}
egenskaber. Hermed menes de love, der g\ae{}lder for operatorerne,
og som vi kan illusterer med f\o{}lgende eksempel. Hvis \verb"+" er
almindelig addition af heltal, er der ingen der undrer sig over
f\o{}lgende udtryk\index{tvetydighed}
$$ \mbox{\verb"x + y + z"} $$
uanset, at det faktisk er {\em tvetydigt}, da det kan opfattes 
som \verb"x + (y + z)" eller som \verb"(x + y) + z". Grunden til at
vi ikke bliver forvirrede er, at det er ligegyldigt, hvilket af de
to alternativer vi v\ae{}lger, fordi \verb"+" opfylder den
{\em associative} lov. Denne lov udtrykker netop at de to alternativer
er lige gode, det vil sige, for alle \verb"x", \verb"y" og \verb"z"
g\ae{}lder
$$ \mbox{\verb"x + (y + z) = (x + y) + z"} $$
Udover associativitet benytter vi os ogs\aa{} af de {\em kommutative\/}
og {\em distributive\/} love for heltalsaritmetik. Mere pr\ae{}cist
g\ae{}lder der for addition og multiplikation af heltal
f\o{}lgende love\index{associativ}\index{kommutativ}\index{distributiv}
\begin{center}
\begin{tabular}{ll}
Associativitet: & \verb"x + (y + z) = (x + y) + z" \\
                & \verb"x * (y * z) = (x * y) * z" \\
Kommutativitet: & \verb"x + y = y + x"\\
                & \verb"x * y = y * x"\\
Distributivitet: & \verb"x * (y + z) = (x * y) + (x * z)"
\end{tabular}
\end{center}
Det interessante er nu, at hvis \verb"+" og \verb"*" i stedet betyder
union og join af relationer, s\aa{} g\ae{}lder ovenst\aa{}ende
fem love stadig. Ikke nok med det, der g\ae{}lder yderligere
f\o{}lgende sammenh\ae{}nge mellem union, join, select og project
(hvor vi antager, at begge sider af ligningerne er lovlige udtryk)
\begin{center}
\begin{tabular}{rcl}
 \verb"(x * y) ? (b)" & \verb"=" & \verb"(x ? (b)) * (y ? (b))"\\
 \verb"(x + y) ? (b)" & \verb"=" & \verb"(x ? (b)) + (y ? (b))"\\
 \verb"(x + y) |+ a" & \verb"=" & \verb"(x |+ a) + (y |+ a)"\\
 \verb"(x |+ a) ? (b)" & \verb"=" & \verb"(x ? (b)) |+ a"
\end{tabular}
\end{center}
Hertil kommer, at select spiller sammen med de logiske operatorer
p\aa{} f\o{}lgende m\aa{}de
\begin{center}
\begin{tabular}{rcl}
\verb"x ? (b and c)" & \verb"=" & \verb"(x ? (b)) * (x ? (c))"\\
\verb"x ? (b or c)" & \verb"=" & \verb"(x ? (b)) + (x ? (c))"\\
\verb"x ? (not b)" & \verb"=" & \verb"x - (x ? (b))"
\end{tabular}
\end{center}
Der findes andre, mere subtile love, som vi ikke skal komme n\ae{}rmere
ind p\aa{} her, men som sammen med ovenst\aa{}ende fx kan anvendes til
effektivitetsfremmende omskrivninger.

Betragt som et simpelt eksempel herp\aa{} den distributive lov for
union og project
\begin{center}
\verb"(x + y) |+ a = (x |+ a) + (y |+ a)"
\end{center}
Vi antager, at \verb"x" og \verb"y" hver indeholder ca.\ 10.000 tupler,
men at \verb"a"-attribut\-ten ikke indeholder mere end 100 forskellige
v\ae{}rdier. Ved at udregne venstresiden skal der f\o{}rst skabes en
relation \verb"(x + y)" med ca.\ 20.000 tupler, som derefter
skal projiceres ned til en relation med kun ca.\ 100 tupler.
Ved en udregning af h\o{}jresiden derimod nedsk\ae{}res \verb"x" og
\verb"y" udmiddelbart til beskeden st\o{}rrelse, hvorefter
foreningsm\ae{}ngden beregnes. Ved at bruge h\o{}jresiden undg\aa{}r
vi alts\aa{} at konstruere et unyttigt mellemresultat med ca.\
20.000 tupler. Der findes andre eksempler, som viser at man kan opn\aa{}
endnu mere dramatiske gevinster ved hj\ae{}lp af passende omskrivninger
af denne slags.

Til sidst betragter vi igen udtryk, hvor \verb"+" og \verb"*" angiver de
s\ae{}dvanlige aritmetiske operatorer. De f\ae{}rreste vil have
problemer med at forst\aa{} f\o{}lgende udtryk
\begin{center}
\verb"x + y * z"
\end{center}
selv om det igen kan opfattes p\aa{} to forskellige m\aa{}der, nemlig
\verb"x + (y * z)" og \verb"(x + y) * z". N\aa{}r vi automatisk opfatter
det f\o{}rste af disse alternativer som det korrekte, er det fordi vi 
ved, at \verb"*" if\o{}lge almindelig vedt\ae{}gt {\em binder h\aa{}rdere\/}
end \verb"+". I {\sc Rasmus} findes der ligeledes s\aa{}danne
bindings\-reg\-ler (eller {\em prioriteter}) \index{prioritet}
for alle operatorer. I nedenst\aa{}ende prioriterede
liste af operatorer
binder de \o{}verste h\aa{}rdest og de nederste svagest.
\begin{center}
\verb"not  |+  |-  !"\\
\verb"and  *  /  mod"\\
\verb"or  +  -"\\
\verb"?  :"\\
\verb"=  <>  <  >  <=  >="
\end{center}
Dette betyder, at f\o{}lgende (tvetydige) venstresider skal opfattes som
angivet af de tilsvarende h\o{}jresider
\begin{center}
\begin{tabular}{rcl}
\verb"x * y |+ a" &\verb"="& \verb"x * (y |+ a)"\\
\verb"x + y ? b" &\verb"="& \verb"(x + y) ? b"\\
\verb"b and c or d" &\verb"="& \verb"(b and c) or d"\\
\verb"b or c < d" &\verb"="& \verb"(b or c) < d"\\
\verb"x or y = z and w" &\verb"="& \verb"(x or y) = (z and w)"
\end{tabular}
\end{center}
Det er ikke sikkert, at disse (og alle de \o{}vrige) implikationer
af operatorprioriteterne er i overensstemmelse med s\ae{}dvanlig
intuition. Det kan derfor normalt 
anbefales, at man s\ae{}tter de tilt\ae{}nkte paranteser eksplicit.

\newpage
\section{Opdateringer og Rapporter}
I realistiske databaser er det ikke altid tilstr\ae{}kkeligt
at beregne nye relationer udfra gamle. Man skal ogs\aa{} have mulighed for
at opdatere tilstanden og for at udskrive rapporter i mere generelle
formater.

\subsection{Tilstands\ae{}ndringer}
I {\sc Rasmus} kan man \ae{}ndre den aktuelle tilstand ved hj\ae{}lp af 
udtrykket
\begin{center}
{\bf id} \verb":=" {\bf udtryk}
\end{center}
Resultatet er det samme som for {\bf udtryk}, men beregningen har
den {\em sideeffekt}, at navnet {\bf id} \ae{}ndres til at have denne
v\ae{}rdi. N\aa{}r udtryk p\aa{} denne m\aa{}de kan for\aa{}rsage tilstands\ae{}ndringer, s\aa{}
giver det mening at udvide syntaksen med f\o{}lgende konstruktion
\begin{center}
{\bf udtryk$_1$} \verb";" {\bf udtryk$_2$}
\end{center}
der f\o{}rst beregner {\bf udtryk$_1$} og derefter {\bf udtryk$_2$},
der giver det endelige resultat. Det f\o{}rste udtryk beregnes s\aa{}ledes
kun fordi det har en sideeffekt.

\eksempel{
F\o{}lgende funktion fjerner holdet \verb"hold" fra \verb"Runde1" og \verb"Runde2"
\begin{center}
\begin{tabular}{l}
\verb"func (hold:Text) -> (Text)"\\
\SS{2}\verb"Runde1:=(Runde1?(\#.HjHold<>hold) and (\#.UdeHold<>hold));"\\
\SS{2}\verb"Runde2:=(Runde2?(\#.HjHold<>hold) and (\#.UdeHold<>hold));"\\
\SS{2}\verb'hold++" er nu fjernet"'\\
\verb"end"
\end{tabular}
\end{center}
}
        

%% \subsection{Udskrifter}
%% {\sc Rasmus} underst\o{}tter en simpel mekanisme til at udskrive 
%% rapporter p\aa{} tekstfiler. Udtrykket 
%% \begin{center}
%% \verb"open("{\bf tekstudtryk}\verb")"
%% \end{center}
%% \aa{}bner UNIX-filen med navn {\bf tekstudtryk}, der herefter er den aktuelle
%% fil; udtrykket
%% \begin{center}
%% \verb"write("{\bf tekstudtryk}\verb")"
%% \end{center}
%% skriver {\bf tekstudtryk} i den aktuelle fil; linjeskift angives med
%% \verb'"'$\backslash$\verb'n"'; udtrykket 
%% \begin{center}
%% \verb"close"
%% \end{center}
%% lukker den aktuelle fil, s\aa{} dens indhold bliver permanent.

%% \eksempel{
%% F\o{}lgende udtryk skriver en lille historie p\aa{} en fil
%% \begin{center}
%% \begin{tabular}{l}
%% {\tt open("r\o{}dh\ae{}tte");}\\
%% {\tt write("Den lille R\o{}dh\ae{}tte gik ud i skoven,$\backslash$n");}\\
%% {\tt write("og s\aa{} smed de ulven ned i br\o{}nden.$\backslash$n");}\\
%% {\tt close}
%% \end{tabular}
%% \end{center}
%% }

%% \subsection{Sorterede genneml\o{}b}
%% I forbindelse med relationer har vi flere gange understreget, at
%% {\em r\ae{}kkef\o{}lgen\/} af tuplerne ikke har nogen betydning. Det er dog
%% ikke sandt for udskrifter, hvor man ofte gerne vil have oplysninger
%% i fx alfabetisk orden. For at afhj\ae{}lpe dette problem findes der
%% to varianter af operatoren {\em forall\/} (og af {\em factor}).
%% Udtrykkene 
%% \begin{center}
%% \verb"!<("{\bf relation}\verb")"... og \verb"!>("{\bf relation}\verb")"...
%% \end{center}
%% fungerer ligesom den almindelige {\em forall\/} eller {\em factor},
%% bortset fra, at tup\-lerne v\ae{}lges sorteret efter de angivne attributter.
%% Varianten \verb"!<" sorterer i ikke-aftagende orden og varianten
%% \verb"!>" sorterer i ikke-voksende orden. For almindelige relationelle
%% udtryk er en s\aa{}dan r\ae{}kkef\o{}lge ikke relevant, men n\aa{}r man genererer udskrifter,
%% s\aa{} kan den v\ae{}re meget nyttigt.

%% \eksempel{
%% Dette udtryk udskriver p\aa{} filen \verb"holdliste"
%% de deltagende fodboldhold i alfabetisk orden
%% \begin{center}
%% \begin{tabular}{l}
%% \verb'open("holdliste");'\\
%% \verb'write("De deltagende fodboldhold er:$\backslash$n$\backslash$n");'\\
%% \verb'(!<(Kampe)|HjHold: (write(\#.HjHold); write("$\backslash$n")));'\\
%% \verb'close'
%% \end{tabular}
%% \end{center}
%% }

%% \newpage
%% \addcontentsline{toc}{section}{Opgaver}
%% \subsection*{Opgave R1}
%% Betragt f\o{}lgende relationer
%% $$ X = 
%% \begin{array}{|c|c|}
%% \hline
%% A & B \\ \hline
%% 1 & 3 \\
%% 3 & 4 \\ \hline
%% \end{array}
%% \;\;\;
%% Y =
%% \begin{array}{|c|c|}
%% \hline
%% A & B \\ \hline
%% 1 & 3 \\
%% 5 & 7 \\ \hline
%% \end{array}
%% \;\;\;
%% Z = 
%% \begin{array}{|c|c|}
%% \hline
%% B & C \\ \hline
%% 3 & 7 \\
%% 3 & 8 \\ 
%% 4 & 9 \\\hline
%% \end{array}
%% $$
%% Beregn (med h\aa{}ndkraft) f\o{}lgende udtryk
%% \begin{itemize}
%% \item[a)] $X$ \verb"+" $Y$
%% \item[b)] $X$ \verb"-" $Y$
%% \item[c)] $X$ \verb"*" $Z$
%% \item[d)] $Y$ \verb"*" $Z$
%% \item[e)] $X$ \verb"*" $Y$
%% \item[f)] $Z$ \verb"|+" $B$
%% \item[g)] ($Z$ \verb"|+" $B$) \verb"*" ($Z$ \verb"|+" $C$)
%% \item[h)] $X$ \verb"*" $Y$ \verb"*" $Z$
%% \end{itemize}

%% \subsection*{Opgave R2}
%% Besvar f\o{}lgende sp\o{}rgsm\aa{}l om materialet i \verb"eksempler/fodbold90".
%% \begin{itemize}
%% \item[a)] P\aa{} hvilke datoer blev der spillet uafgjort?
%% \item[b)] Hvilke hold har vundet alle deres hjemmekampe?
%% \item[c)] Hvilke hold har tabt til Br\o{}ndby?
%% \item[d)] Hvilke hold har vundet 2--1?
%% \item[e)] Hvilke par af hold har gensidigt vundet over hinanden?
%% \end{itemize}

%% \subsection*{Opgave R3}
%% Hvilke af f\o{}lgende udtryk er lovlige? Hvad er skemaerne af
%% resultaterne af de lovlige udtryk?
%% \begin{itemize}
%% \item[a)] \verb"(Kampe * Resultater * Tips) |- HjHold"
%% \item[b)] \verb"(Resultater + Runde2) - Runde1"
%% \item[c)] \verb"Tips[Uge <- Dato] * Spilledag"
%% \item[d)] \verb"Tips[Uge <- Dato] * ((Spilledag |- Dato)[Kamp <- Dato])"
%% \end{itemize}

%% \subsection*{Opgave R4}
%% Forklar, hvad f\o{}lgende udtryk beregner
%% \begin{verbatim}
%%      (+ val A = Runde1 + Runde2
%%         val B = A * Resultater
%%         val C = B ? (#.HjHold = "AGF")
%%         val D = B ? (#.UdeHold = "AGF")
%%         val E = C ? (#.HjScore > #.UdeScore)
%%         val F = D ? (#.UdeScore > #.HjScore)
%%         val G = E + F
%%         val H = G[Id <- Kamp]
%%         val I = H * Spilledag
%%         in I |+ Dato
%%      +)
%% \end{verbatim}

%% \subsection*{Opgave R5}
%% Lav tipskuponen for uge 20 og 21 for materialet
%% i \verb"eksempler/fodbold90". Dokument\'{e}r, hvordan du gjorde det.
%% Skriv dem ud sorteret p\aa{} \verb"Nr"-attributterne.
%% Beskriv, hvordan man givet en tipskupon og en {\em tipsr\ae{}kke\/}
%% i form af en relation med skema
%% {\tt
%% \begin{center}
%% \begin{tabular}{|c|c|}
%% \hline
%% Nr:Int & Tegn:Text \\
%% \hline
%% \end{tabular}
%% \end{center}
%% }
%% kan beregne hvor mange rigtige tegn, der er p\aa{} r\ae{}kken.

%% \subsection*{Opgave R6}
%% Beskriv, hvordan tilstanden \ae{}ndrer sig under beregningen af
%% f\o{}lgende udtryk
%% \begin{verbatim}
%%    (+ val A = 3
%%       val B = (+ val C = A+2
%%                  val D = C*A
%%                  in D-1
%%               +)
%%       val C = (+ val B = B+1
%%                  val E = (+ Val E = B-A
%%                             in B+E
%%                          +)
%%                  in B+E
%%               +)
%%       in (+ val D = A+B-C in 2*D +) + 3
%%     +)
%% \end{verbatim}

%% \subsection*{Opgave R7}
%% En {\em ruterelation\/} er af formen
%% {\tt
%% \begin{center}
%% \begin{tabular}{|l|l|}
%% \hline fra & til \\\hline
%% London & Paris\\
%% Tirstrup & London\\
%% Paris & New York\\
%% Paris & Moskva \\
%% London & Orlando\\
%% \hline
%% \end{tabular}
%% \end{center}
%% }
%% Den angiver de ruter, som et flyselskab tilbyder.
%% \begin{itemize}
%% \item[a)] Skriv et udtryk, der beregner den {\em symmetriske lukning\/}
%% af en ruterelation, det vil sige, s\o{}rger for, at hvis der er en rute fra
%% $A$ til $B$, s\aa{} er der ogs\aa{} en rute fra $B$ til $A$.
%% \item[b)] Skriv et udtryk, der beregner den {\em refleksive lukning\/}
%% af en ruterelation, det vil sige, s\o{}rger for, at der for alle
%% byer $A$ er en rute fra $A$ til $A$ (som man m\aa{} formode er ret billig).
%% \item[c)] Lad \verb"SAS" og \verb"KLM" v\ae{}re to ruterelationer. Hvad
%% beregner udtrykket
%% $$ \mbox{\verb"((SAS[til <- via])*(KLM[fra <- via])) |- via"} $$
%% \end{itemize}

%% \subsection*{Opgave R8}
%% Skriv en funktion, der givet navnet p\aa{} et hold beregner en relation med en
%% enkelt attribut {\tt Overm\ae{}nd} af type \verb"Text", der indeholder
%% navnene p\aa{} de hold, der har besejret det angivne hold.

%% \subsection*{Opgave R9}
%% Ved hj\ae{}lp af {\em forall\/} og tupeludtryk kan man efterligne
%% visse af de andre relationsoperatorer. Antag, at \verb"X" er en
%% relation med skema
%% {\tt
%% \begin{center}
%% \begin{tabular}{|c|c|c|c|}
%% \hline
%% a:Int & b:Int & c:Int & d:Text \\
%% \hline
%% \end{tabular}
%% \end{center}
%% }
%% Skriv {\em forall\/} udtryk, der giver samme resultat som
%% \begin{itemize}
%% \item[a)] \verb"X[a<-aa,c<-cc]"
%% \item[b)] \verb"X |+ b,d"
%% \end{itemize}

%% \subsection*{Opgave R10}
%% En attribut er en {\em n\o{}gle}, hvis dens v\ae{}rdi entydigt
%% bestemmer de \o{}vrige attributter, eller med andre ord, hvis
%% der ikke er to forskellige tupler med samme n\o{}glev\ae{}rdi.
%% I fodboldmaterialet er fx \verb"Id" og \verb"Kamp" n\o{}gler.
%% \begin{itemize}
%% \item[a)] Forklar hvorledes udtrykket
%% \begin{small}
%% \begin{verbatim}
%% not(rel(tup(x:true))<=(!(R):rel(tup(x:|rel(tup(A:#.A))*R|>1))))
%% \end{verbatim}
%% \end{small}
%% afg\o{}r om attributten \verb"A" er en n\o{}gle i relationen \verb"R".
%% \item[b)] Kan du skrive et simplere udtryk med samme effekt?
%% \end{itemize}

%% \subsection*{Opgave R11}
%% {\em Det Cartesiske produkt\/} (eller {\em krydsproduktet\/}) af
%% to m\ae{}ngder $A$ og $B$ defineres som
%% $$ A\times B = \{(a,b) \,|\, a\in A\;\wedge\;b\in B\} $$
%% For $n$ m\ae{}ngder $A_1,\ldots,A_n$ definerer vi
%% $$ A_1\times\cdots\times A_n = \{(a_1,\ldots,a_n)\,|\,a_1\in A_1\;\wedge
%% \ldots \wedge a_n\in A_n\} $$
%% \begin{itemize}
%% \item[a)] Lad $X=\{a,b\}$ og $Y=\{b,c,d,e\}$. Angiv $X\times Y$, $X\times X$
%% og $X\times X\times Y$.
%% \item[b)] Hvor mange elementer er der i $A_1\times\cdots\times A_n$?
%% \end{itemize}
%% {\em Potensm\ae{}ngden\/} af en m\ae{}ngde $A$ defineres som
%% $$ {\cal P}(A) = \{ S\,|\,S\subseteq A\}$$
%% det vil sige m\ae{}ngden af alle delm\ae{}ngder af $A$, inklusive
%% $\{\}$ og $A$.
%% \begin{itemize}
%% \item[c)] Lad $Z=\{a,b,c\}$. Angiv ${\cal P}(Z)$ og ${\cal P}(\{\})$.
%% \item[d)] Hvor mange elementer er der i ${\cal P}(A)$?
%% \end{itemize}
%% En {\em (matematisk) relation\/}
%% over m\ae{}ngderne $A_1,\ldots,A_n$ defineres
%% som et element i ${\cal P}(A_1\times\cdots\times A_n)$. Hvad har
%% dette med {\sc Rasmus} relationer at g\o{}re?

%% \subsection*{Opgave R12}
%% En relation med f\o{}lgende skema indeholder oplysninger om
%% afleveringsopgaver p\aa{} dProg1:
%% {\tt
%% \begin{center}
%% \begin{tabular}{|r|r|r|}
%% \hline
%% \AA{}rskort:Int & Uge:Int & Godkendt:Bool \\
%% \hline
%% \end{tabular}
%% \end{center}
%% }
%% For hver student og hver uge angives det, om den p\aa{}g\ae{}ldende opgave
%% er godkendt. Der skal skrives en funktion \verb"OK" med hovede
%% \begin{center}
%% {\tt func (\aa{}rskort:Int) -> (Bool)}
%% \end{center}
%% der afg\o{}r, om den angivne students obligatoriske forl\o{}b er godkendt.
%% Reg\-len er som bekendt, at mindst 8 opgaver skal godkendes, heraf skal
%% mindst 5 v\ae{}re i ulige uger, og opgaven for uge 49 skal altid v\ae{}re
%% god\-kendt.

%% \subsection*{Opgave R13}
%% I \verb"Spilledag" relationen skrives den 16.\ marts som
%% \verb"316" og den 1.\ april som \verb"401". Skriv en funktion
%% {\tt P\ae{}nDato} med hovede
%% \begin{center}
%% {\tt func (dato:Text) -> (Text)}
%% \end{center}
%% der \ae{}ndrer formatet fra \verb:"316": til \verb:"den 16. marts":. Brug
%% {\tt P\ae{}nDato} til at g\o{}re \verb"Spilledag" relationen p\ae{}nere.

%% \subsection*{Opgave R14}
%% En telefonbog kan repr\ae{}senteres som en relation med skema
%% {\tt
%% \begin{center}
%% \begin{tabular}{|r|r|}
%% \hline
%% Navn:Text & Nummer:Int\\
%% \hline
%% \end{tabular}
%% \end{center}
%% }
%% Skriv f\o{}lgende seks funktioner; \verb"Init" med hovede
%% \begin{center}
%% {\tt func () -> (Rel)}
%% \end{center}
%% der giver en tom telefonbog; \verb"Defined" med hovede
%% \begin{center}
%% {\tt func (TB:Rel, navn:Text) -> (Bool)}
%% \end{center}
%% der fort\ae{}ller, om \verb"navn" forekommer i telefonbogen \verb"TB";
%% \verb"Dom" med hovede
%% \begin{center}
%% {\tt func (TB:Rel) -> (Rel)}
%% \end{center}
%% der giver en relation med de navne, der forekommer i telefonbogen \verb"TB";
%% \verb"Lookup" med hovede
%% \begin{center}
%% {\tt func (TB:Rel, navn:Text) -> (Int)}
%% \end{center}
%% der giver nummeret p\aa{} \verb"navn", der skal findes i telefonbogen \verb"TB";
%% \verb"Update" med hovede
%% \begin{center}
%% {\tt func (TB:Rel, navn:Text, nummer:Int) -> (Rel)}
%% \end{center}
%% der giver en udvidet kopi af telefonbogen \verb"TB", i hvilken 
%% \verb"navn" har \verb"nummer"; hvis \verb"navn" allerede findes i
%% \verb"TB", s\aa{} skal nummeret blot \ae{}ndres;
%% endeligt er der funktionen \verb"Delete" med hovede
%% \begin{center}
%% {\tt func (TB:Rel, navn:Text) -> (Rel)}
%% \end{center}
%% der giver en kopi af telefonbogen \verb"TB" i hvilken \verb"navn" er
%% fjernet.

%% \subsection*{Opgave R15}
%% Skriv et udtryk, der ud fra en relation med skema 
%% {\tt
%% \begin{center}
%% \begin{tabular}{|l|l|}
%% \hline
%% Hold:Text & Points:Int\\
%% \hline
%% \end{tabular}
%% \end{center}
%% }
%% (som p\aa{} side 37) finder de tre bedste hold i turneringen.

%% \subsection*{Opgave R16}
%% I \verb"eksempler/astronomi" findes en relation \verb"planeter" med skema
%% {\tt
%% \begin{center}
%% \begin{tabular}{|l|l|r|r|r|}
%% \hline
%% Planet & Navn & Afstand & Diameter & Opdaget \\
%% \hline
%% \end{tabular}
%% \end{center}
%% }
%% der indeholder oplysninger om planeter, deres m\aa{}ner, m\aa{}nens
%% afstand fra planeten, m\aa{}nens diameter, og \aa{}rstallet hvor m\aa{}nen
%% blev opdaget (af mennesker). Skriv og udf\o{}r {\sc Rasmus} udtryk, der
%% beregner
%% \begin{itemize}
%% \item[a)] en relation med navnene p\aa{} de planeter, der har st\o{}rre
%% m\aa{}ner end vores.
%% \item[b)] en relation med navnene p\aa{} de planeter, der i \aa{}r 1850 ikke
%% havde kendte m\aa{}ner.
%% \item[c)] en relation med samme skema som \verb"planeter", men som for
%% hver planet kun indeholder tuplet med dennes st\o{}rste m\aa{}ne.
%% \end{itemize}

%% \subsection*{Opgave R17}
%% I \verb"eksempler/benzin" ligger en relation \verb"forbrug" med skema
%% {\tt
%% \begin{center}
%% \begin{tabular}{|l|l|l|l|}
%% \hline
%% Dato:Text & Km:Int & Volumen:Int & Pris:Int\\
%% \hline
%% \end{tabular}
%% \end{center}
%% }
%% der angiver benzinforbruget for en bil (med en pedantisk ejer).
%% Hvert tupel beskriver en optankning med dato, bilens kilometert\ae{}ller,
%% volumen af indk\o{}bt benzin (i centiliter) og prisen (i \o{}rer).
%% Skriv {\sc Rasmus} udtryk, der beregner
%% \begin{itemize}
%% \item[a)] den gennemsnitlige pris betalt pr.\ liter benzin i perioden.
%% \item[b)] den dato, hvor benzinen var dyrest.
%% \item[c)] en funktion, der givet et \aa{}rstal (som tekst: \verb:"88":,
%% \verb:"89":, etc.)
%% beregner, hvor mange km bilen har k\o{}rt i dette \aa{}r.
%% \item[d)] hvor sikker ejeren er p\aa{} h\aa{}nden; det vil i denne
%% sammenh\ae{}ng sige, hvor t\ae{}t han i gennemsnit er p\aa{} at
%% tanke et helt antal liter.
%% \end{itemize}

%% \subsection*{Opgave R18}
%% Betragt igen ruterelationerne fra opgave R7. Med {\em produktet\/}
%% \verb"R"$\cdot$\verb"S" mener vi resultatet af
%% \begin{center}
%% \verb"((R[til <- via])*(S[fra <- via])) |- via"
%% \end{center}
%% Vi forkorter \verb"R"$^1$ = \verb"R" og \verb"R"$^n$ =
%% \verb"R"$\cdot$\verb"R"$^{n-1}$ for $n>1$. 
%% Den {\em transitive lukning\/} af 
%% \verb"R" defineres som 
%% $$ \mbox{\tt R}^+ = \sum_{i=1}^{\infty} \mbox{\tt R}^i =
%% \mbox{\tt R} \mbox{\tt +} \mbox{\tt R}^2 \mbox{\tt +} \mbox{\tt R}^3 \mbox{\tt +}
%% \mbox{\tt R}^4 \mbox{\tt +} \ldots$$
%% \begin{itemize}
%% \item[a)] Hvilke oplysninger om ruter indeholder \verb"R"$^+$?
%% \item[b)] Argument\'{e}r for, at der for alle \verb"R" findes et
%% $n\geq 1$, s\aa{}dan at
%% $$ \mbox{\tt R}^+ = \sum_{i=1}^n \mbox{\tt R}^i$$
%% \end{itemize}
%% Betragt f\o{}lgende udtryk
%% \begin{verbatim}
%% (+ val Prod  = func (S,R: Rel) -> (Rel)
%%                  ((S[til <- via])*(R[fra <- via])) |- via
%%                end
%%    val Trans = func (X: Rel) -> (Rel)
%%                  (+ val P = Prod(X,R)+R
%%                     in if P=X -> X
%%                         & true -> Trans(P)
%%                        fi
%%                  +)
%%                end
%%    in Trans(R)
%% +)
%% \end{verbatim}
%% \begin{itemize}
%% \item[c)] Argument\'{e}r for, at udtrykket beregner \verb"R"$^+$.
%% \end{itemize}

%% \subsection*{Opgave R19}
%% Betragt en relation \verb"Tilmeldinger" med skema
%% {\tt
%% \begin{center}
%% \begin{tabular}{|r|r|}
%% \hline
%% \AA{}rskort:Text & Kursus:Text\\
%% \hline
%% \end{tabular}
%% \end{center}
%% }
%% der angiver, hvilke studenter, der \o{}nsker at f\o{}lge hvilke kurser.
%% Til brug for skemaplanl\ae{}gning \o{}nsker vi en relation \verb"Konflikter"
%% med skema
%% {\tt
%% \begin{center}
%% \begin{tabular}{|r|r|l|}
%% \hline
%% Kursus1:Text & Kursus2:Text & Antal:Int\\
%% \hline
%% \end{tabular}
%% \end{center}
%% }
%% hvor tuplerne angiver, at blandt de tilmeldte til \verb"Kursus1" er der
%% \verb"Antal" studenter, der ogs\aa{} har tilmeldt sig \verb"Kursus2".
%% Skriv et udtryk, der foretager denne beregning (vink: factor).

%% \subsection*{Opgave R20}
%% Afg\o{}r hvilke af f\o{}lgende algebraiske love, der g\ae{}lder for
%% {\sc Rasmus} udtryk. Giv en \underline{\em kort\/}
%% begrundelse for hver lov, der g\ae{}lder,
%% og angiv et modeksempel for hver lov, der {\em ikke\/}
%% g\ae{}lder. Som i noten kan det antages, at begge sider af ligningerne
%% er lovlige udtryk.
%% \begin{itemize}
%% \item[a)] \verb"(x * y) |+ a = (x |+ a) * (y |+ a)"
%% \item[b)] \verb"(x * x) = x"
%% \item[c)] \verb"(x |+ a) * (x |- a) = x"
%% \item[d)] \verb"(x ? b) + (x ? not b) = x"
%% \item[e)] \verb"x - (y - z) = (x - y) + z"
%% \item[f)] \verb"(x |- a) - (y |- a) = (x - y) |- a"
%% \item[g)] \verb"x ? (a and not b) = (x ? a) - (x ? b)"
%% \end{itemize}

%% \subsection*{Opgave R21}
%% Modific\'{e}r opgave R5, s\aa{}ledes at tipskuponen udskrives i det
%% velkendte format p\aa{} en tekstfil.

\newpage
%\printindex
\end{document}
